\documentclass[12pt,a4paper,dvipdfmx,enablejfam=true]{masterthesis}
\usepackage{ascmac}
\usepackage[dvipdfmx]{graphicx}
\usepackage{here}
\usepackage{comment}
\usepackage{url}
\usepackage{multirow}
\usepackage{setspace}
\usepackage{amsmath}
\usepackage{fancyhdr}
\usepackage{siunitx}
\usepackage{amsfonts,amsthm,amssymb}
\usepackage{mathtools}
\usepackage{titlesec}
\usepackage{tocloft}

% \chapter のフォントを変更
\titleformat{\chapter}[display]
  {\normalfont\huge\bfseries}{\chaptertitlename\ \thechapter\ 章}{30pt}{\Huge}
% \section のフォントを変更
\titleformat{\section}
  {\normalfont\Large\bfseries}{\thesection}{1em}{}
% \subsection のフォントを変更
\titleformat{\subsection}
  {\normalfont\large\bfseries}{\thesubsection}{1em}{}
%  目次のフォント変更,「第○章」と章タイトルがかぶるため手動で外す
\renewcommand\cftchapnumwidth{8.5ex}
% \renewcommand{\cftchapfont}{\normalfont\bfseries}{1em}{}

%%%%%%%%%  表紙  %%%%%%%%%
\title{Unity アセットの投稿と QR コード閲覧を備えた HMD 向け AR プレゼンテーションシステム ARShow}
\author{閻 永祥}
\adviser{赤嶺 有平}
\frontmatter %ページ番号をromanに変更
\etitle{ARShow: A AR Presentation System with Unity Assets Upload and QR Codes Access for HMD}
\eauthor{YONGXIANG YAN}
\eadviser{Prof. Yuhei Akamine}
\shusa{國田 樹}
\hukusaa{副査1}
\hukusab{副査2}

%%%%%%%%%  ドキュメント開始  %%%%%%%%%
\begin{document}
\maketitle
% 概要
\chapter*{概要}

本論文では、3D モデルと QR コードを統合し、HMD (ヘッドマウントディスプレイ)を通じて AR (拡張現実)空間での高度な可視化を実現するプレゼンテーションシステム「ARShowNode」と「ARShow」の開発、設計、およびその有用性について詳述する。AR 技術は現在、教育、マーケティング、娯楽など多岐にわたる分野で活用されているが、アクセシビリティとインタラクティブ性の高度な両立が依然として課題となっている。本研究では、制作者向けの「ARShowNode」プロジェクトと閲覧者向けの「ARShow」プロジェクトからなる双方向的なプラットフォームを提案し、Unity エンジンを基盤とした技術的仕様、アセットのアップロードとダウンロード、および 3D モデルの可視化プロセスについて論じる 。

先行研究である「AR MUSE」などは、Android 端末を用いて 3D モデルと QR コードを紐付ける安価かつユーザーフレンドリーな解決策を提示した。これに対し、本研究の独創性は、出力デバイスとして HMD を採用することで没入感を飛躍的に向上させ、さらに 3D モデルと共に動的スクリプトを同時配信して実行可能にした点にある。これにより、従来の静的な展示に留まらない、複雑な挙動を伴うインタラクティブな AR 表現を閲覧者に提供することが可能となった。

システムの評価にあたっては、Unity 開発経験を有する制作者および一般の閲覧者を対象とした被験者実験を行い、システムユーザビリティの観点から定量的および定性的な分析を実施した。その結果、本研究で提案したシステムは高度な AR 展示の構築において、アクセシビリティに優れ、かつ表現力豊かな実用的ソリューションであることが実証された。

% 概要(英)
\chapter*{Abstruct}

This thesis discusses the development, design, and efficacy of ARShow, a presentation system for Head-Mounted Displays (HMDs) that facilitates the integration of 3D models with QR codes to enable advanced augmented reality (AR) visualization. While AR technology is currently employed across diverse fields such as education, marketing, and entertainment, the simultaneous achievement of accessibility and high-level interactivity remains a persistent challenge. This research proposes a bidirectional platform comprising a "curation side" for creators and a "viewer side" for audiences, detailing technical specifications based on the Unity engine as well as the processes for asset importation, exportation, and 3D model visualization.

The predecessor research, AR MUSE, offered an affordable and user-friendly solution for linking 3D models to QR codes via Android devices. In contrast, the novelty of the present study lies in the adoption of HMDs as the output medium to drastically enhance immersion, as well as the implementation of simultaneous distribution for "dynamic scripts" alongside 3D models. This allows for the delivery of interactive AR experiences involving complex behaviors that extend beyond conventional static exhibits.

The system was evaluated through subject testing involving both creators with Unity development experience and general viewers, with quantitative and qualitative analyses performed using the System Usability Scale (SUS). The results indicate that ARShow serves as an accessible, highly expressive, and practical solution for the construction of advanced AR exhibitions. This work aims to establish AR as a more accessible and powerful presentation methodology for creators utilizing 3D models globally.

% 研究関連業績
\chapter*{研究関連論文業績}
\begin{itemize}
\item 閻永祥,赤嶺有平,``HMD と QR コードを活用した文化財展示向けの AR システム'' 第24回情報科学技術フォーラム, K-003, E棟304, pp.495-498, 2025.
\end{itemize}


\mainmatter %ページ番号をarabicに変更

% 目次(subsubsectionまで反映させる
\setcounter{tocdepth}{2}
\tableofcontents
\newpage
% 図目次
\listoffigures
\newpage
% 表目次
\listoftables

%%%%%%%%%  本文  %%%%%%%%%
\chapter{はじめに}
\section{背景と目的}

近年、博物館や美術館をはじめとする多様な文化施設において、デジタル技術を導入した展示手法が急速に普及している。特に、AR(拡張現実)技術を用いた空間芸術展示は、物理的な制約を超えた情報の提示や、現実空間とデジタルコンテンツが融合する新たな視覚体験を提供する手段として、その重要性を増している。

こうした展示環境の変化に伴い、コンテンツの制作および運用を担う主体の役割も変容している。従来の博物館展示において、キュレーター(学芸員)の主たる役割は資料の収集・保存・調査研究であった。しかし、メディアアートやインタラクティブな展示が増加する現代においては、施設空間を活用しつつ、Unity 等のゲームエンジンを用いて自ら AR コンテンツの実装やインタラクション設計を行う、エンジニアリング能力を有したアーティストや展示製作者がその中心となりつつある。そこで本研究では、便宜上こうした「技術的背景を持ち、展示の企画から実装までを担う人物」をキュレーターと定義し、一般的な博物館法に基づく学芸員とは区別して論じるものとする。

これらエンジニアリング能力を持つキュレーターの活動は、展示表現の可能性を広げる一方で、実際の運用現場においては依然として大きな課題を抱えている。

第一の課題は、展示運営におけるコミュニケーションコストの増大である。AR 空間芸術のような専門性の高い展示では、鑑賞者が自身のデバイス等で体験を行う際、キュレーターが一人ひとりに対しアプリケーションの導入方法や操作手順を詳細に説明する必要が生じる場合が多い。鑑賞者にとっても、展示ごとに異なる独自の操作体系をその都度理解することは負担であり、これが円滑な鑑賞体験を阻害する要因となっている。

第二の課題は、コンテンツの更新性と配布プロセスの硬直性である。本研究が対象とするキュレーターは Unity 開発に習熟しているものの、iOS、Android、あるいは Meta Quest 等の HMD 向けビルドにおいては、セキュリティやプラットフォームの制約上、実行バイナリ自体の更新なしにプログラムの挙動(C\# スクリプト)を変更することは極めて困難である。そのため、展示期間中に演出の微調整やロジックの修正が必要になった場合でも、アプリケーション全体の再ビルドと再配布を余儀なくされる。特に展示会場において、Quest 端末等の HMD に対して再インストールを行う作業は、PC 接続や開発者モードでの操作を要し、1 回の更新に多大な時間を要するため、複数台のデバイスを運用する展示現場では現実的ではない。したがって、アプリ本体を更新せずに展示ロジックのみを更新可能なホットアップデート機構の確立は、展示の質と鮮度を維持するために不可欠な要素である。

本研究の目的は、上述した課題を解決するため、HMD (ヘッドマウントディスプレイ)と QR コードを活用し、制作者としてのキュレーターと閲覧者としての鑑賞者の間を媒介する中間的なアプリケーション基盤「ARShow」を提案および構築することである(図\ref{fig:arshow_arch})。

\begin{figure}[htbp]
\begin{center}
\includegraphics[width=100mm]{./figs/はじめに/ARShowシステムアーキテクチャ.png}
\caption[ARShowシステムアーキテクチャ]{ARShowシステムアーキテクチャ.}
\label{fig:arshow_arch}
\end{center}
\end{figure}

具体的には、以下の三つの目標を達成することを目指す。

\begin{itemize}
    \item \textbf{第一}: Unity 向けフレームワークである HybridCLR を導入することで、従来困難であったスクリプトレベルでのコードホットアップデートを実現する。これにより、キュレーターは「ARShowTool」を通じて、自身の制作した 3D モデルや複雑なインタラクションロジックを、再ビルドの手間なく迅速かつ低コストで配信可能な環境を構築する。
    \item \textbf{第二}: QR コードをスキャンするだけで即座に体験を開始できる導入の簡便さに加え、HMD 特有の物理コントローラを排除したハンドトラッキングによる身体的操作や、音声コマンドによる自然言語入力を統合する。これらマルチモーダルなインタラクションの実現により、鑑賞者の学習コストを最小化し、誰もが直感的に参加できる没入感の高い AR 体験を提供する。
    \item \textbf{第三}: 単なる機能の提案に留まらず、Unity 上での制作、サーバへの配信、そしてクライアントアプリでの動的ロードという一連のワークフローが、実際の展示運用に耐えうるシステムとしての有効性とユーザビリティを明らかにする。
\end{itemize}

本研究を通じて、展示内容の更新がアプリケーションの再配布に依存しない柔軟な運用環境を実現する。これにより、開発スキルを持つキュレーターにとっては表現の自由度と制作効率を高め、鑑賞者にとっては手軽で没入感のある体験を提供し、新たなデジタル展示のプラットフォームを確立することを目指す。

\section{論文の構成}

本論文の構成は以下の通りである.

第 2 章 基礎概念を述べる.

第 3 章 関連研究を述べる.

第 4 章 提案システムを詳細に説明する.

第 5 章 評価方針を立てる.

第 6 章 総括と今後の課題を述べる.

\chapter{基礎概念}

本章では、本研究が提案するシステム設計および実装の基盤となる概念と技術的背景について述べる。まず、展示の主体であるキュレーターの役割の変化と、デジタル展示における課題を整理する。次に、その解決手段としての AR (拡張現実)技術の定義と分類について概説する。続いて、実装環境である Unity および Meta XR SDK の特性を述べ、本研究の核心技術である HybridCLR を用いたホットアップデート機構、およびシステム全体の運用を支える通信アーキテクチャについて詳述する。

\section{キュレーションの変遷と定義}

\subsection{博物館における伝統的役割}

伝統的にキュレーター(学芸員)は、博物館法や ICOM (国際博物館会議)の規定に基づき、資料の収集、保存、調査研究、および展示企画を専門的に担う職種として定義されてきた。博物館という制度的枠組みの中で、歴史的かつ芸術的価値を持つ資料を体系化し、教育的配慮のもとで公衆へ提示することがその主たる役割であった。

\subsection{インディペンデントキュレーターの台頭}

近年、特定の博物館組織に所属せず、独自の文脈とテーマ設定によって展覧会を構成するインディペンデントキュレーターの活動が顕著となっている。Obrist ら[1]が指摘するように、現代のキュレーターの役割は単なる管理や保存から、新たな意味を創出するプロデューサーとしての側面を強めている。彼らの活動領域は物理空間にとどまらず、デジタル技術を用いた空間表現にも及んでおり、展示空間そのものの再定義を行っている。

\subsection{本研究における定義と課題}

AR を用いた空間芸術展示において、キュレーターの役割は鑑賞体験全体の設計者へと拡張されている。しかし、高度なデジタル技術の導入は、鑑賞者に対するデバイス操作説明やアプリケーション導入支援といった、展示の本質とは異なるコミュニケーションコストの増大を招いているのが現状である。
本研究では、Unity 等の技術的背景を持つか否かに関わらず、デジタル空間で展示構成を行い、鑑賞者へ体験を提供する主体として「キュレーター」を定義する。その上で、技術的な障壁を取り除き、彼らが表現活動に専念できる環境の構築を目指す。

\section{拡張現実技術}

\subsection{拡張現実の定義}

AR (Augmented Reality)とは、実世界の情報にコンピュータ生成情報をリアルタイムに重畳し、人間の知覚を拡張する技術である。Azuma [2]による定義では、以下の 3 要素を満たすものとされる。

\begin{itemize}
    \item 現実と仮想の結合(Combines real and virtual)
    \item リアルタイムなインタラクション(Interactive in real time)
    \item 三次元的な位置合わせ(Registered in 3D)
\end{itemize}

\subsection{ロケーションベース AR}

ロケーションベース AR は、GPS (全地球測位システム)や磁気センサ、加速度センサ等の位置情報を利用し、特定の地理的座標にデジタルコンテンツを配置する手法である(図\ref{fig:location_based_ar})。
本手法はPokemon GOに代表されるような広域な屋外展示には適している。しかし、屋内においては GPS 信号の遮断により位置特定精度が著しく低下することや、高さ方向の正確な整合(レジストレーション)に課題が残る場合が多く、ミリ単位の配置精度が求められる精密な芸術作品の展示には不向きである。

% ロケーションベースARの図
\begin{figure}[htbp]
\begin{center}
\includegraphics[width=80mm]{./figs/基礎概念/ロケーションベースAR.png}
\caption{ロケーションベースARの概念図 [5]}
\label{fig:location_based_ar}
\end{center}
\end{figure}

\subsection{マーカ型ビジョンベース AR}

マーカ型ビジョンベース AR は、特定の画像(マーカ)をカメラで認識し、その特徴量に基づいてデジタルコンテンツの表示位置や傾きを決定する手法である(図\ref{fig:marker_based_ar})。
本研究では、AR コンテンツの識別子(ID)と空間的な配置基準点(空間アンカー)の両方の機能を併せ持つ QR コードをマーカとして採用する。QR コードを用いることで、画像認識の安定性が向上し、鑑賞者は意図した作品を正確な位置座標で呼び出すことが可能となる。

% マーカ型ビジョンベースARの図
\begin{figure}[htbp]
\begin{center}
\includegraphics[width=80mm]{./figs/基礎概念/マーカ型ビジョンベースAR.png}
\caption{マーカ型ビジョンベースARの概念図 [5]}
\label{fig:marker_based_ar}
\end{center}
\end{figure}

\subsection{マーカレス型ビジョンベース AR}

マーカレス型ビジョンベース AR は、特定のマーカを必要とせず、SLAM (Simultaneous Localization and Mapping)技術等を用いて周囲の環境形状をリアルタイムに解析し認識する手法である(図\ref{fig:markerless_ar})。
本研究で使用する HMD (ヘッドマウントディスプレイ)である Meta Quest 3 等の最新デバイスでは、深度センサとカメラを用いた高度な空間認識が可能であり、壁面や床面を物理的な制約としてデジタルコンテンツに反映させることができる。しかし、この方式は環境特徴点の抽出と追跡に多大な計算リソースを要するため、モバイル HMD 単体での動作においては、マーカ型と比較してリアルタイム性および長時間稼働時の安定性に課題が残る。

% マーカレス型ビジョンベースARの図
\begin{figure}[htbp]
\begin{center}
\includegraphics[width=80mm]{./figs/基礎概念/マーカレス型ビジョンベースAR.png}
\caption{マーカレス型ビジョンベースARの概念図 [5]}
\label{fig:markerless_ar}
\end{center}
\end{figure}

\section{Unity 開発プラットフォーム}

\subsection{Unity (ユニティ)}

Unity は Unity Technologies 社[3]が提供するリアルタイム 3D 開発プラットフォームであり、現在の XR コンテンツ開発におけるデファクトスタンダードである。物理演算、レンダリング、オーディオ処理などの機能が統合されたゲームエンジンであり、C\# スクリプトによる柔軟なロジック記述が可能であることから、ゲーム産業のみならず建築、自動車、学術研究など多岐にわたる分野で利用されている。

\subsection{Prefab (プレハブ)}

Prefab は GameObject、コンポーネント(C\# スクリプト)、およびプロパティ設定を一つのアセットとしてテンプレート化する機能である。これにより、展示作品を構成する複雑なオブジェクト群(3Dモデル、テクスチャ、アニメーション、挙動スクリプト等)を一つの単位として管理し、実行時に動的に生成(インスタンス化)または破棄することが容易となる。

\subsection{AssetBundle (アセットバンドル)}

AssetBundle は Unity のアセットを実行時に外部からロード可能な形式でアーカイブする機能である。本研究では、1 つの展示作品を 1 つの AssetBundle に対応させる設計を採用している。これにより、アプリケーション本体(バイナリ)を更新することなく、サーバ上のアーカイブファイルを差し替えるだけで、コンテンツの追加や更新を行うことが可能となる。

\subsection{MonoBehaviour}

MonoBehaviour は、Unity におけるすべてのスクリプトコンポーネントが継承すべき基底クラスである。本クラスを継承することで、スクリプトは GameObject にアタッチ可能なコンポーネントとして機能し、Unity のイベントライフサイクル(Start, Update, OnDestroy 等)にフックされる。
本研究において、各展示作品の固有の振る舞い(アニメーション制御やインタラクション処理)はすべて MonoBehaviour を継承した C\# クラスとして実装される。これにより、Unity のインスペクタ上でのパラメータ調整が可能となり、開発効率とメンテナンス性が担保される。

\subsection{UUID (GUID)}

Unity はプロジェクト内のアセット(ファイル)を管理するために、ファイルパスではなく、UUID (Universally Unique Identifier)、Unity 上では一般に GUID (Global Unique Identifier)と呼ばれる一意の識別子を使用する。

ファイル名やディレクトリ構造が変更された場合でも、GUID が維持されている限り、Unity はアセット間の参照関係(例えば、Prefab がどのテクスチャを使用しているか等)を正しく解決できる。本研究のホットアップデート機構においても、サーバから取得したリソースとローカルのスクリプトを正しくリンクさせるために、この一意性が重要な役割を果たす。

\subsection{Assembly}

Assembly (アセンブリ)とは、C\# コードがコンパイルされた後のバイナリ単位(.dll)を指す。Unity では通常、ユーザーが記述したスクリプトは 「Assembly-CSharp.dll」 という単一のアセンブリにコンパイルされる。
しかし、大規模な開発や本研究のような動的な機能追加を行う場合、コードを機能ごとに分割し、独自の Assembly Definition (アセンブリ定義)を作成して管理することが推奨される。後述する HybridCLR は、このアセンブリ単位でのロードと実行制御を行うことで、スクリプトのホットアップデートを実現している。

\subsection{Metaファイル}

Unity プロジェクト内のすべてのアセットファイルには、対となる 「.meta」 ファイルが自動生成される。このメタファイルには、前述の GUID や、アセットごとのインポート設定(テクスチャの圧縮形式やモデルのスケール設定等)が記録されている。

バージョン管理システムを利用する際や、外部からアセットを取り込む際には、このメタファイルを正しく同期させる必要がある。メタファイルの欠損や不整合は参照切れ(Missing Reference)を引き起こし、アプリケーションの動作不全に直結するためである。

\subsection{制約と課題}

標準的な Unity の仕様において、AssetBundle はテクスチャやモデルデータなどの非コードアセットの更新には適しているが、コンパイル済みのロジック(C\# スクリプト)の更新を含めることはできない。これは、インタラクションの挙動を変更したい場合にアプリ全体の再ビルドとストアへの再配布が必要となることを意味し、展示運営上の大きなボトルネックとなっていた。本研究では、後述する HybridCLR を導入することでこの制約を突破する。

\section{Meta XR All-in-One SDK}

\subsection{パススルー機能}

Meta XR SDK は、Meta Quest シリーズのハードウェア機能を Unity 上で制御するための開発キットである。特に本研究では、外部カメラで取得した現実映像に CG を合成するカラーパススルー機能を活用する。これにより、現実空間と展示コンテンツがシームレスに融合した AR (Mixed Reality)体験を構築し、鑑賞者が現実の展示会場の文脈を失うことなく作品を鑑賞できる環境を提供する。

\subsection{Meta XR Interaction SDK}

Meta XR Interaction SDK は、ハンドトラッキングやコントローラ操作を抽象化し、標準的なインタラクションを提供するライブラリである。本 SDK を用いることで、開発者はハードウェアごとの差異を意識することなく実装が可能となる。また、鑑賞者は「掴む(Grab)」「指差す(Poke)」といった直感的な身体動作で AR コンテンツを操作することが可能となり、没入感を阻害しない自然な操作体系(NUI: Natural User Interface)が実現される。

\subsection{Meta XR Voice SDK}

Meta XR Voice SDK は音声認識モジュールであり、アプリケーションに対して音声入力インターフェースを提供する。本 SDK は、マイクから取得した音声データの正規化やストリーミング処理を担い、後述する自然言語処理プラットフォーム Wit.AI との通信を仲介する。これにより、HMD 装着時のハンズフリー操作や、コントローラでは表現しきれない「作品解説の呼び出し」「シーン切り替え」といった抽象度の高いコマンドを、直感的な音声入力によって実装することが可能となる。

\subsection{Wit.AI}

Wit.AI は Meta 社が提供する自然言語処理(Natural Language Processing)プラットフォームであり、ユーザーの非構造化データ(音声やテキスト)をコンピュータが処理可能な構造化データへと変換するクラウドサービスである。本研究において、Wit.AI は鑑賞者の発話意図を解析し、具体的な操作命令へと変換する核心的なエンジンとして機能する。Wit.AI の処理プロセスは図\ref{fig:witai_process}に示すように、主に以下の要素によって構成される。

% Wit.AIの処理プロセスの図
\begin{figure}[htbp]
\begin{center}
\includegraphics[width=80mm]{./figs/基礎概念/Wit.AIの処理プロセス.png}
\caption{Wit.AIにおける音声処理プロセス [5]}
\label{fig:witai_process}
\end{center}
\end{figure}

\begin{itemize}
    \item \textbf{Intent (インテント)}: ユーザーの発話が「何をしようとしているのか」という意図を定義したものである。例えば、「解説を再生して」や「次の作品へ移動」といった発話に対し、それぞれ \texttt{PlayDescription} や \texttt{MoveToNext} といった識別子を割り当てる。システムは返却されたインテント識別子に基づき、実行すべき C\# メソッドを決定する。
    \item \textbf{Entity (エンティティ)}: 発話に含まれる具体的なパラメータや変数を抽出するための定義である。例えば、「作品Aを見せて」という発話において、「見せて」はインテントであるが、「作品A」は操作対象を特定する重要な変数である。Wit.AI は事前に学習させたキーワードや文脈に基づき、このような固有名称を抽出し、引数としてアプリケーションへ返却する。
    \item \textbf{Trait (トレイト)}: 発話全体の意味合いやニュアンスを分類する機能である。肯定(Yes)または否定(No)の判定や感情分析などに用いられる。これにより、確認ダイアログに対する応答判定などが容易になり、より自然な対話フローの構築が可能となる。
\end{itemize}

これらの機能により、Wit.AI は単なる音声のテキスト化(Speech-to-Text)にとどまらず、文脈理解(Natural Language Understanding)を伴う高度なインタラクションを実現する。

\section{HybridCLR とホットアップデート}

\subsection{IL2CPP の技術的課題}

Unity の標準的なビルド方式である IL2CPP (Intermediate Language to C++)モードでは、C\# コードがビルド時に C++ へ事前コンパイル(AOT: Ahead-Of-Time)される。この仕組みにより、実行速度の向上やセキュリティの確保が可能となる反面、実行中に新たな C\# コード(アセンブリ)を動的に読み込んで実行することは、メモリ管理や実行権限の構造上、不可能であった。

\subsection{HybridCLR の導入}

HybridCLR [4]は、この IL2CPP 環境下において、AOT 実行とインタープリタ実行を混在させることで、C\# の完全なホットアップデートを実現する革新的なフレームワークである。HybridCLR は、IL2CPP のランタイムを拡張し、コンパイル済みのネイティブバイナリ上で、サーバからダウンロードした DLL (アセンブリ)内のメタデータと IL 命令を直接解釈・実行するインタプリタ機能を提供する。

\subsection{AddComponent ベースの更新}

これは、実行時(ランタイム)にロードしたアセンブリ内のクラス情報を利用し、\texttt{AddComponent} メソッドを通じて既存の GameObject に新たなコンポーネント(ロジック)を動的に付与する手法である。これにより、アプリリリース時には存在しなかった全く新しい機能を作品に追加できる。本研究ではこの技術を採用することで、キュレーターが作成した複雑なインタラクションロジックを、AssetBundle として鑑賞者アプリへ即座に反映させることを可能にしている。

\subsection{Asset ベースの更新}

これは、Prefab 内にシリアライズされたスクリプト参照を、実行時にロードした最新のスクリプトコードへと自動的に解決(リマップ)する手法である。
ただし、技術的な注意点として、スクリプトをデシリアライズする際には、生成時と同一のメタデータ構造を保持している必要がある。そのため、異なる Unity プロジェクトで生成された Prefab は、GUID の不一致等により中身のスクリプトを自動的にリマップできない場合がある。本研究では、この制約を考慮したアセット管理フローを構築している。

\section{静的ファイル配信サーバ}

本システムにおけるサーバは、動的なプログラム処理(アプリケーションロジック)を持たず、クライアントからのリクエストに応じてファイル(映像、3Dモデル、AssetBundle 等)を配信することに特化した静的ファイルサーバである。
実装には ASP.NET Core を採用している。これにより、サーバサイドとクライアント(Unity)間で C\# 言語による統一的なデータ定義が可能となり、保守性を向上させている。

\section{システムアーキテクチャ}

提案システムはクライアント・サーバモデルを採用し、以下の 3 つの主要コンポーネントから構成される(図\ref{fig:proposed_system})。

% 提案システム構成の図
\begin{figure}[htbp]
\begin{center}
\includegraphics[width=80mm]{./figs/基礎概念/提案システム構成.png}
\caption{提案システムの全体構成図 [5].}
\label{fig:proposed_system}
\end{center}
\end{figure}

\begin{itemize}
    \item \textbf{ARShowNode}: AR 展示物の作成とアップロードを行うキュレーター側プログラム
    \item \textbf{ARShow}: 閲覧者用クライアントアプリケーション
    \item \textbf{Static File Server}: コンテンツを配信する静的ファイルサーバ
\end{itemize}

この 3 つの部分が有機的に連携することにより、キュレーターは自身のコンテンツを容易に配信かつ更新でき、閲覧者は最新の展示物をシームレスに体験できる環境が実現される。

展示作品の実体である AssetBundle は、アプリケーション内部には保持されず、外部サーバに配置される。クライアントアプリは QR コードの読み取りをトリガーとして、必要なデータのみをオンデマンドで取得し展開する。このアーキテクチャにより、鑑賞者のデバイスストレージを圧迫することなく、作品数を無制限に拡張することが可能となる。また、展示内容の更新はサーバ側のアーカイブファイル差し替えのみで完結するため、前述した「更新プロセスの困難さ」および「コミュニケーションコスト」の課題を根本から解決する基盤となる。


参考文献 (References)
[1] Obrist, H. U. (2014). Ways of Curating. Faber \& Faber.
[2] Azuma, R. T. (1997). A Survey of Augmented Reality. Presence: Teleoperators and Virtual Environments, 6(4), 355-385.
[3] Unity Technologies. (2024). Unity User Manual. https://docs.unity3d.com/
[4] HybridCLR. (2024). HybridCLR Documentation. https://github.com/focus-creative-games/hybridclr

\chapter{関連研究}
\label{exprmnt:physarum}
\section{目的}


\subsection{粘菌の培養方法}

\subsection{生体信号の解析方法}

\chapter{提案システム}
\label{exprmnt:physarum}

\section{設計思想}

本節では、本研究で提案したAR展示システムの設計思想について述べる。

本システムは、AR(拡張現実)を用いた芸術展示において、キュレーター(制作者)と鑑賞者(閲覧者)の間に存在するコミュニケーションコストの削減、および展示コンテンツの持続可能な運用を解決することを主たる目的として設計された。従来のデジタル芸術展示、特にアプリベースのAR展示においては、策展者が作品を公開するためにアプリケーションそのものを配布する必要があり、更新のたびに再配布や再インストールが求められるという課題があった。また、展示会終了とともにアプリのサポートが終了し、作品の鑑賞が不可能になるケースも散見される。さらに、鑑賞者にとっても、展示ごとに異なる操作方法や導入手順を学習する必要があり、これが鑑賞体験への障壁となっていた。

これらの課題に対し、本システムはキュレーターと鑑賞者を媒介するプラットフォームとしての役割を果たす。本システムの設計思想を図\ref{fig:tobita_config}に示す。具体的には、UnityのAssetBundle機能とHybridCLRによるホットアップデート技術を組み合わせることで、アプリケーション本体を更新することなく、展示コンテンツ(ロジックを含む)をサーバ経由で動的に追加や更新可能なアーキテクチャを採用した。策展者は、提供されるツールキットに従いAR展品(Node)を作成し、サーバへアップロードするだけで、恒久的なアクセスが可能となる。一方、鑑賞者は単一のクライアントアプリ(ARShow)を使用し、会場のQRコードをスキャンするだけで、即座に該当する作品コンテンツをダウンロードし、鑑賞を開始することができる。この設計により、ハードウェアやアプリの更新に依存しない作品の永続性と、直感的かつ統一された鑑賞体験の両立を目指している。

\clearpage
% 設計思想の図
\begin{figure}[htbp]
\begin{center}
\includegraphics[width=80mm]{./figs/提案システム/設計思想.png}
\caption[設計思想]{設計思想 [5].}
\label{fig:tobita_config}
\end{center}
\end{figure}

\section{開発環境}

本節では、本システムの構築および動作検証に使用したハードウェアおよびソフトウェア環境について述べる。

\subsection{ハードウェア}

本システムの開発および実行には、コンテンツ制作とサーバホスティングを行うPCと、AR体験を提供するHMD(Head Mounted Display)を用いた。

開発用PC(図\ref{fig:pc})には、Windows 11 Professional (25h2) を搭載したワークステーションを使用した。Windows環境を採用した主な理由は、Meta Quest Link機能を利用することで、実機へのビルドしてインストールを行うことなく、Unityエディタ上で直接AR空間の動作確認が可能となり、開発効率が著しく向上するためである。

% PCの図
\begin{figure}[htbp]
\centering
\includegraphics[width=80mm]{./figs/提案システム/ハードウェア/PC.png}
\caption[PC]{開発に使用したPCの外観}
\label{fig:pc}
\end{figure}

本研究で使用したPCの具体的なハードウェア構成を表\ref{table:pc_spec}に示す。

% PCスペック表
\begin{table}[htbp]
  \caption[開発用PCの仕様]
  {開発用PC (Redmi G Pro 2024) のハードウェア構成}
  \label{table:pc_spec}
  \centering
  \begin{tabular}{lll}
  \hline
  \multicolumn{1}{|l||}{分類} & \multicolumn{1}{l|}{項目} & \multicolumn{1}{l|}{仕様} \\ \hline \hline
  \multicolumn{1}{|l||}{General}  & \multicolumn{1}{l|}{Model Name}  & \multicolumn{1}{l|}{Redmi G Pro 2024}  \\ \hline
  \multicolumn{1}{|l||}{System}  & \multicolumn{1}{l|}{OS}  & \multicolumn{1}{l|}{Windows 11 Professional}  \\ \hline
  \multicolumn{1}{|l||}{Hardware} & \multicolumn{1}{l|}{CPU} & \multicolumn{1}{l|}{Intel Core i7-14650HX} \\ \hline
  \multicolumn{1}{|l||}{Hardware}  & \multicolumn{1}{l|}{GPU}  & \multicolumn{1}{l|}{NVIDIA GeForce RTX 4060 Laptop}  \\ \hline
  \multicolumn{1}{|l||}{Hardware}  & \multicolumn{1}{l|}{Memory (RAM)}  & \multicolumn{1}{l|}{32 GB (DDR5)}  \\ \hline
  \multicolumn{1}{|l||}{Hardware}  & \multicolumn{1}{l|}{Storage}  & \multicolumn{1}{l|}{1 TB SSD (PCIe 4.0)}  \\ \hline
  \end{tabular}
\end{table}

鑑賞用デバイスには、ビデオシースルー機能を備えたスタンドアロン型HMDであるMeta Quest 3(図\ref{fig:quest3})を採用した。また、開発PC機とHMDの接続には、大容量データの高速転送および安定したストリーミングを実現するため、転送速度2.5Gbps以上の帯域を持つUSB-Cケーブル(図\ref{fig:cable})を使用した。

\begin{figure}[htbp]
\centering
% Meta Quest 3の図
\begin{minipage}{0.48\textwidth}
  \centering
  \includegraphics[width=\linewidth]{./figs/提案システム/ハードウェア/MetaQuest3.png}
  \caption[MetaQuest3]{MetaQuest3 [5].}
  \label{fig:quest3}
\end{minipage}
\hfill
% USB-Cケーブルの図
\begin{minipage}{0.48\textwidth}
  \centering
  \includegraphics[width=\linewidth]{./figs/提案システム/ハードウェア/USBCケーブル.png}
  \caption[USBCケーブル]{USBCケーブル [5].}
  \label{fig:cable}
\end{minipage}
\end{figure}

ここで、Meta Quest 3の主要仕様を表\ref{table:quest3_spec}にまとめる。

\clearpage
% Meta Quest 3 のスペック表
\begin{table}[htbp]
  \caption[Meta Quest 3の仕様]
  {実験に使用したHMD (Meta Quest 3) の主な仕様}
  \label{table:quest3_spec}
  \centering
  \begin{tabular}{lll}
  \hline
  \multicolumn{1}{|l||}{分類} & \multicolumn{1}{l|}{項目} & \multicolumn{1}{l|}{仕様} \\ \hline \hline
  \multicolumn{1}{|l||}{General}  & \multicolumn{1}{l|}{Model}  & \multicolumn{1}{l|}{Meta Quest 3}  \\ \hline
  \multicolumn{1}{|l||}{Processor}  & \multicolumn{1}{l|}{SoC}  & \multicolumn{1}{l|}{Snapdragon XR2 Gen 2}  \\ \hline
  \multicolumn{1}{|l||}{Display} & \multicolumn{1}{l|}{Resolution} & \multicolumn{1}{l|}{2064 $\times$ 2208 pixels per eye} \\ \hline
  \multicolumn{1}{|l||}{Display}  & \multicolumn{1}{l|}{Refresh Rate}  & \multicolumn{1}{l|}{90 Hz / 120 Hz}  \\ \hline
  \multicolumn{1}{|l||}{Optics}  & \multicolumn{1}{l|}{Field of View}  & \multicolumn{1}{l|}{110$^\circ$ (H) / 96$^\circ$ (V)}  \\ \hline
  \multicolumn{1}{|l||}{Memory}  & \multicolumn{1}{l|}{RAM}  & \multicolumn{1}{l|}{8 GB}  \\ \hline
  \multicolumn{1}{|l||}{Camera}  & \multicolumn{1}{l|}{Passthrough}  & \multicolumn{1}{l|}{Full-color (4 MP, 18 PPD)}  \\ \hline
  \end{tabular}
\end{table}

本研究で言及したUSB-Cケーブルの主要仕様を表\ref{table:cable_spec}にまとめる。

% ケーブルのスペック表
\begin{table}[htbp]
  \caption[USBケーブルの仕様]
  {HMDとPCの接続に使用したUSBケーブルの仕様}
  \label{table:cable_spec}
  \centering
  \begin{tabular}{lll}
  \hline
  \multicolumn{1}{|l||}{分類} & \multicolumn{1}{l|}{項目} & \multicolumn{1}{l|}{仕様} \\ \hline \hline
  \multicolumn{1}{|l||}{Interface}  & \multicolumn{1}{l|}{Connector Type}  & \multicolumn{1}{l|}{USB Type-C to Type-C}  \\ \hline
  \multicolumn{1}{|l||}{Standard}  & \multicolumn{1}{l|}{USB Standard}  & \multicolumn{1}{l|}{USB 3.2 Gen 1}  \\ \hline
  \multicolumn{1}{|l||}{Performance} & \multicolumn{1}{l|}{Max Transfer Rate} & \multicolumn{1}{l|}{5 Gbps} \\ \hline
  \multicolumn{1}{|l||}{Physical}  & \multicolumn{1}{l|}{Length}  & \multicolumn{1}{l|}{5.0 m}  \\ \hline
  \multicolumn{1}{|l||}{Material}  & \multicolumn{1}{l|}{Core Type}  & \multicolumn{1}{l|}{Optical Fiber}  \\ \hline
  \end{tabular}
\end{table}

ネットワーク環境として、本研究ではローカルエリアネットワーク(LAN)内での運用を想定し、スマートフォン(図\ref{fig:smartphone})のテザリング機能を用いてPCとHMDを同一ネットワークに接続した。なお、スマートフォン再起動等によるIPアドレスの変更に対応するため、開発機側で固定IP設定もしくは動的IPの確認手順を確立して運用した。

\clearpage
% スマートフォンの図
\begin{figure}[htbp]
\centering
\includegraphics[width=80mm]{./figs/提案システム/ハードウェア/スマートフォン.png}
\caption[スマートフォン]{スマートフォン [5].}
\label{fig:smartphone}
\end{figure}

本研究で使用したスマートフォンの主な仕様を表\ref{table:smartphone_spec}に示す。

% スマートフォンのスペック表
\begin{table}[htbp]
  \caption[スマートフォンの仕様]
  {使用したスマートフォンの仕様一覧}
  \label{table:smartphone_spec}
  \centering
  \begin{tabular}{lll}
  \hline
  \multicolumn{1}{|l||}{分類} & \multicolumn{1}{l|}{項目} & \multicolumn{1}{l|}{仕様} \\ \hline \hline
  \multicolumn{1}{|l||}{Device}  & \multicolumn{1}{l|}{Model Name}  & \multicolumn{1}{l|}{Google Pixel 7a}  \\ \hline
  \multicolumn{1}{|l||}{System}  & \multicolumn{1}{l|}{OS Version}  & \multicolumn{1}{l|}{Android 13}  \\ \hline
  \multicolumn{1}{|l||}{Network} & \multicolumn{1}{l|}{Wi-Fi Standard} & \multicolumn{1}{l|}{IEEE 802.11ax (Wi-Fi 6E)} \\ \hline
  \multicolumn{1}{|l||}{Network}  & \multicolumn{1}{l|}{Tethering Band}  & \multicolumn{1}{l|}{5 GHz / 2.4 GHz}  \\ \hline
  \multicolumn{1}{|l||}{Hardware}  & \multicolumn{1}{l|}{Processor (SoC)}  & \multicolumn{1}{l|}{Google Tensor G2}  \\ \hline
  \multicolumn{1}{|l||}{Hardware}  & \multicolumn{1}{l|}{Memory (RAM)}  & \multicolumn{1}{l|}{8 GB}  \\ \hline
  \end{tabular}
\end{table}

\subsection{ソフトウェア}

本システムのソフトウェア開発環境は、Unityを統合開発環境の中核とし、Meta社が提供するXR開発キットおよびサードパーティ製のライブラリ群によって構成されている。

主要な開発プラットフォームとしてUnityを使用し、Meta Quest 3のパススルー機能や空間認識機能を利用するためにMeta XR All-in-One SDKおよびOpenXRプラグインを導入した。これにより、高精度なパススルー機能や空間アンカー、ハンドトラッキングといった固有機能へのアクセスを可能にしている。同時に、標準規格であるOpenXR Pluginを併用することで、将来的なデバイス互換性と拡張性を担保した。ユーザーインターフェース(UI)の構築においては、Meta Horizon OS UI Set(図\ref{fig:uiset})を採用した。Questのシステム標準UIと親和性の高いデザインコンポーネントを使用することで、ユーザーに対して違和感のない、直感的な操作体験を提供することを目的としている。

% UI Setの図
\begin{figure}[htbp]
\centering
\includegraphics[width=80mm]{./figs/提案システム/ソフトウェア/MetaHorizonOSUISet.png}
\caption[MetaHorizonOSUISet]{MetaHorizonOSUISet [5].}
\label{fig:uiset}
\end{figure}

また、本システムのアーキテクチャ上の特徴として、動的なコード実行(ホットアップデート)機能の実装が挙げられる。通常のUnity IL2CPPビルドでは困難な、実行時のロジック更新を実現するため、C\#インタープリタフレームワークであるHybridCLRを組み込んだ。これにより、実機への再インストールを行うことなく機能の拡張が可能となり、開発サイクルの大幅な効率化を実現している。その他、外部入力機能として、QRコード認識には軽量かつ高速なZXing.unityを、音声コマンド認識にはMeta Wit.aiを活用した。

本システムの開発および実装に使用した主要なプラグインとライブラリ一覧を表\ref{table:dev_tools_spec}に示す。

% Unityとツールの表
\begin{table}[htbp]
  \caption[開発環境およびライブラリ構成]
  {本システムで使用した開発環境およびライブラリのバージョン一覧}
  \label{table:dev_tools_spec}
  \centering
  \begin{tabular}{lll}
  \hline
  \multicolumn{1}{|l||}{分類} & \multicolumn{1}{l|}{名称} & \multicolumn{1}{l|}{バージョン} \\ \hline \hline
  \multicolumn{1}{|l||}{Platform}  & \multicolumn{1}{l|}{Unity}  & \multicolumn{1}{l|}{6000.0.62f1}  \\ \hline
  \multicolumn{1}{|l||}{XR Plugin}  & \multicolumn{1}{l|}{Meta XR All-in-One SDK}  & \multicolumn{1}{l|}{v81}  \\ \hline
  \multicolumn{1}{|l||}{XR Plugin} & \multicolumn{1}{l|}{OpenXR Plugin} & \multicolumn{1}{l|}{v1.1.54} \\ \hline
  \multicolumn{1}{|l||}{Library}  & \multicolumn{1}{l|}{HybridCLR}  & \multicolumn{1}{l|}{v8.5.0}  \\ \hline
  \multicolumn{1}{|l||}{Library}  & \multicolumn{1}{l|}{ZXing.unity}  & \multicolumn{1}{l|}{v3.5.4}  \\ \hline
  \end{tabular}
\end{table}

実機へのデプロイ、パフォーマンス監視、および画面収録などのデバイス管理プロセスには、公式の統合開発ツールであるMeta Quest Developer Hub(図\ref{fig:mqdh})を使用した。また、PC側でのレンダリング検証やUnityエディタのPlay Modeを活用した迅速なデバッグを行うため、Meta Horizon Link(図\ref{fig:link})による有線接続環境を構築した。

\clearpage
\begin{figure}[htbp]
\centering
% MQDHの図
\begin{minipage}{0.48\textwidth}
  \centering
  \includegraphics[width=\linewidth]{./figs/提案システム/ソフトウェア/MQDH.png}
  \caption[Meta Quest Developer Hub]{Meta Quest Developer Hub [5].}
  \label{fig:mqdh}
\end{minipage}
\hfill
% Linkの図
\begin{minipage}{0.48\textwidth}
  \centering
  \includegraphics[width=\linewidth]{./figs/提案システム/ソフトウェア/MetaHorizonLink.png}
  \caption[Meta Horizon Link]{Meta Horizon Link [5].}
  \label{fig:link}
\end{minipage}
\end{figure}

サーバサイドの実装においては、クライアントサイド(Unity)と主要開発言語をC\#に統一することで、データモデルの共有やコンテキストスイッチの低減を図るため、ASP.NET Coreを採用した。特に、軽量かつ高速なレスポンスが求められるため、Minimal API構成で実装している。

コーディングおよびデバッグ環境には、軽量かつ拡張性に優れたVisual Studio Codeを選定した。Microsoft社が提供するC\# Dev KitおよびUnity拡張機能を導入することで、コード補完やブレークポイントによるデバッグ効率を最大化した。

本システムの開発および実装に使用した主要なソフトウェア開発環境一覧を表\ref{table:software_env}に示す。

% ソフトウェア開発環境の表
\begin{table}[htbp]
  \caption[開発支援ツールおよびサーバ環境]
  {デバッグツール、サーバフレームワークおよびエディタのバージョン一覧}
  \label{table:software_env}
  \centering
  \begin{tabular}{lll}
  \hline
  \multicolumn{1}{|l||}{分類} & \multicolumn{1}{l|}{名称} & \multicolumn{1}{l|}{バージョン} \\ \hline \hline
  \multicolumn{1}{|l||}{Tool}  & \multicolumn{1}{l|}{Meta Quest Developer Hub}  & \multicolumn{1}{l|}{v3.2}  \\ \hline
  \multicolumn{1}{|l||}{Tool}  & \multicolumn{1}{l|}{Meta Horizon Link}  & \multicolumn{1}{l|}{v83.0.0.224.349}  \\ \hline
  \multicolumn{1}{|l||}{Framework} & \multicolumn{1}{l|}{ASP.NET Core} & \multicolumn{1}{l|}{v10.0.1} \\ \hline
  \multicolumn{1}{|l||}{Editor}  & \multicolumn{1}{l|}{Visual Studio Code}  & \multicolumn{1}{l|}{v1.108}  \\ \hline
  \end{tabular}
\end{table}

\section{システム構成}

本節では、提案システムの具体的な構成要素と、それらが連携するアーキテクチャについて詳細に述べる。
提案システムは大きく分けて、
\begin{itemize}
    \item キュレーターが展示コンテンツを制作および出力するためのUnityプロジェクト「ARShowNode」
    \item 鑑賞者が使用する閲覧用アプリケーション「ARShow」
    \item コンテンツを保持かつ配信する「静的ファイルサーバ」
\end{itemize}
の三つ要素から構成される。提案システムの全体構成を図\ref{fig:system_overview}に示す。

% 提案システムの全体構成図
\begin{figure}[htbp]
\centering
\includegraphics[width=80mm]{./figs/提案システム/提案システムの全体構成図.png}
\caption[提案システムの全体構成図]{提案システムの全体構成図 [5].}
\label{fig:system_overview}
\end{figure}

\subsection{ARShowNode}

ARShowNodeは、キュレーター(制作者)向けに提供されるUnityプロジェクトである(図\ref{fig:arshownode_global})。本プロジェクトは、AR展品(Node)を単位(AssetBundle)として管理し、各Nodeには3Dモデル、音声、映像、および動的ロジック(C\#スクリプト)が含まれる。

% ARShowNode Unityプロジェクトの全体図
\begin{figure}[htbp]
\centering
\includegraphics[width=80mm]{./figs/提案システム/ARShowNode/ARShowNodeGlobal.png}
\caption[ARShowNodeGlobal]{ARShowNodeGlobal [5].}
\label{fig:arshownode_global}
\end{figure}

\subsubsection{HybridCLR}

本システムにおける技術的な核心は、HybridCLRの採用にある。通常のUnity製アプリ(IL2CPPビルド)では、ビルド後にC\#スクリプトの挙動を変更することは不可能である。しかし、HybridCLRを導入することで、C\#コードをコンパイルしたDLLをアセットとして扱い、実行時にインタープリタモードでロードして実行することが可能となる(図\ref{fig:hybridclr_tool})。ARShowNodeでは、キュレーターが記述したスクリプトをHybridCLRによってAOT(Ahead-Of-Time)メタデータとホットアップデート用DLLに変換し、これらをAssetBundleに含めることで、アプリ本体の更新を伴わない展示ロジックの配信を実現している。

\begin{figure}[htbp]
\centering
% HybridCLRツールの図
\begin{minipage}{0.48\textwidth}
  \centering
  \includegraphics[width=\linewidth]{./figs/提案システム/ARShowNode/HybridCLRTool.png}
  \caption[HybridCLRTool]{HybridCLRTool [5].}
  \label{fig:hybridclr_tool}
\end{minipage}
\hfill
% 提案されたUnityツールの図
\begin{minipage}{0.48\textwidth}
  \centering
  \includegraphics[width=\linewidth]{./figs/提案システム/ARShowNode/ARShowTool.png}
  \caption[ARShowTool]{ARShowTool [5].}
  \label{fig:arshow_tool}
\end{minipage}
\end{figure}

\subsubsection{制作のための Unity ツール}

本研究では、キュレーターの作業負荷を軽減するため、ARShowNodeプロジェクト内に専用のUnityエディタ拡張ツールを実装した(図\ref{fig:arshow_tool})。このツールはメニューバーからアクセス可能であり、以下の4つの機能を順次実行することで展示データの配信を行う。

\begin{enumerate}
    \item \textbf{DLLの複製}: HybridCLRによって生成されたAOTメタデータのDLLおよびホットアップデートのDLLを、Unityプロジェクト内のAssetsディレクトリへコピーする。複数作品(Node)を同時制作する場合、各Nodeに対応したDLLを複製する。
    \item \textbf{AssetBundleのビルド}: 事前に設定されたラベル(Node0, Node1...等の識別子)に基づき、各作品のリソースとDLLを含んだAssetBundleを生成する。
    \item \textbf{QRコード生成}: 各AssetBundleの識別子(Node名)情報を格納したQRコード画像を自動生成する。
    \item \textbf{アップロード}: 生成されたAssetBundle群を、稼働中の静的ファイルサーバへ一括アップロードする。
\end{enumerate}

\subsubsection{Node}

「Node」は、本システムにおける展示作品の単位であり、1つのAssetBundleに対応している。各Nodeは、展示物の実体であるPrefab、関連する設定ファイル、メディアアセット(画像、音響)、および制御用スクリプトのアセンブリの集合体である。本研究では、図\ref{fig:node_global}に示すように、同一Unityプロジェクト内で複数のNode(Node0, Node1, Node2...)を並行して制作と管理可能な構造としている。

\clearpage
% 三つNodeの全体図
\begin{figure}[htbp]
\centering
\includegraphics[width=80mm]{./figs/提案システム/Node/NodeGlobal.png}
\caption[NodeGlobal]{NodeGlobal [5].}
\label{fig:node_global}
\end{figure}

\subsubsection{エントリポイント (Entry.cs)}

動的にロードされたスクリプトを、実行時(Runtime)に正しくゲームオブジェクトにアタッチし機能させるため、本システムでは「Entry.cs」という規約に基づいたエントリポイントスクリプトを導入した(図\ref{fig:entry_global})。

Entry.csは各Nodeの初期化ロジックを担い、AssetBundleのロード完了後、ARShowアプリ側から明示的に呼び出されることで、Prefabのインスタンス化や依存コンポーネントのセットアップを実行する。実装コードの一部を図\ref{fig:entry_code}に示す。

\begin{figure}[htbp]
\centering
% 三つNodeに分けるEntry.csの全体図
\begin{minipage}{0.48\textwidth}
  \centering
  \includegraphics[width=\linewidth]{./figs/提案システム/ARShowNode/EntryGlobal.png}
  \caption[EntryGlobal]{EntryGlobal [5].}
  \label{fig:entry_global}
\end{minipage}
\hfill
% Entry.csのコードの一部図
\begin{minipage}{0.48\textwidth}
  \centering
  \includegraphics[width=\linewidth]{./figs/提案システム/ARShowNode/EntryCode.png}
  \caption[EntryCode]{EntryCode [5].}
  \label{fig:entry_code}
\end{minipage}
\end{figure}

通常のUnity開発ではInspector上でスクリプトをアタッチするが、別プロジェクトでビルドされたAssetBundle内のスクリプトを復元する場合、GUIDの不整合等により参照が切れる問題がある。これを回避するため、本システムではHybridCLRの推奨手法に基づき、実行時に「GameObject.AddComponent」メソッドを介してスクリプトを動的に付与する方式を採用した。

\subsubsection{AssetBundle}

UnityのAssetBundle機能を利用し、Nodeごとのリソースとコードをパッケージ化している。前述の通り、本システムではAssetBundle内にHybridCLR用のDLLを含める点が特徴である(図\ref{fig:assetbundle_global})。これにより、3Dモデルやテクスチャだけでなく、インタラクション等の仕組みも含めた完全な展示パッケージとして配信される。

% 提案されたUnityツールにより生成したAssetBundleの例
\begin{figure}[htbp]
\centering
\includegraphics[width=80mm]{./figs/提案システム/ARShowNode/AssetBundleGlobal.png}
\caption[AssetBundleGlobal]{AssetBundleGlobal [5].}
\label{fig:assetbundle_global}
\end{figure}

\subsubsection{WitAI}

音声操作を実現するため、Meta社が提供する自然言語処理サービスWit.aiを導入した(図\ref{fig:witai_config})。Node内で音声認識が必要な場合、ユーザーの音声入力はMeta XR Voice SDKを通じてテキスト化され、Wit.aiサーバへ送信される。サーバ上で事前に定義されたインテント(Intent)やエンティティ(Entity)の解析が行われ、その結果に基づいてUnity側の関数がトリガーされる仕組みである。本研究では、中国語による音声コマンド制御の事例を実装した。

% WitAI配置図
\begin{figure}[htbp]
\centering
\includegraphics[width=80mm]{./figs/提案システム/ARShowNode/WitAI配置.png}
\caption[WitAI配置]{WitAI配置 [5].}
\label{fig:witai_config}
\end{figure}

\subsection{ARShow}

ARShowは、Meta Quest 3上で動作する鑑賞者用アプリケーションである(図\ref{fig:arshow_global})。本アプリは、QRコードのスキャン、AssetBundleのダウンロード、および動的コンテンツの再生環境を提供するコンテナとして機能する。

% ARShow Unityプロジェクトの全体図
\begin{figure}[htbp]
\centering
\includegraphics[width=80mm]{./figs/提案システム/ARShow/ARShowGlobal.png}
\caption[ARShowGlobal]{ARShowGlobal [5].}
\label{fig:arshow_global}
\end{figure}

\subsubsection{スキャンモード}

アプリ起動後、ユーザーはUI上の「ScanQr」ボタンを押下することでスキャンモードへ移行する(図\ref{fig:scan_qr_button})。このモードでは、パススルーカメラの映像上にQRコード検出用のガイドが表示され、認識待機状態となる。

% ScanQrボタンの図
\begin{figure}[htbp]
\centering
\includegraphics[width=80mm]{./figs/提案システム/ARShow/ScanQrボタン.png}
\caption[ScanQrボタン]{ScanQrボタン [5].}
\label{fig:scan_qr_button}
\end{figure}

\subsubsection{QR Code検出}

QRコードの認識にはZXing.unityライブラリを使用し、Quest 3のカメラ映像からフレームごとの解析を行う。有効なQRコード(AssetBundleの識別子)が検出されると、スキャンモードを終了し、コンテンツのロード処理へ移行する。同一のQRコードを再スキャンした場合は、重複ロードを避けるためコンソールへ警告を表示する等の制御を行っている。

\subsubsection{アセットロード}

識別子に基づき、サーバから該当するAssetBundleをダウンロードする。ダウンロード中は、QRコードの物理位置にプログレスバー(進捗率)を空間表示し、ユーザーへのフィードバックを行う。初AssetBundleのダウンロードプログレスが図\ref{fig:node0_progress}に示され、後続のものが図\ref{fig:node2_progress}に示される。

\begin{figure}[htbp]
\centering
% 初のAssetBundleダウンロードプログレスの図
\begin{minipage}{0.48\textwidth}
  \centering
  \includegraphics[width=\linewidth]{./figs/提案システム/Node/Node0Progress.png}
  \caption[Node0Progress]{Node0Progress [5].}
  \label{fig:node0_progress}
\end{minipage}
\hfill
% 後続のAssetBundleダウンロードプログレスの図
\begin{minipage}{0.48\textwidth}
  \centering
  \includegraphics[width=\linewidth]{./figs/提案システム/Node/Node2Progress.png}
  \caption[Node2Progress]{Node2Progress [5].}
  \label{fig:node2_progress}
\end{minipage}
\end{figure}

ダウンロード完了後、システムは以下の手順でアセットを展開する。

\begin{enumerate}
    \item AssetBundleからHybridCLR用のAOTメタデータDLLをメモリ上に展開し、IL2CPPランタイムに登録する。
    \item ホットアップデート用DLLをロードする。
    \item ロードされたアセンブリ内から Entry.cs のエントリポイントメソッドをリフレクションを用いて呼び出す。
\end{enumerate}

これにより、図\ref{fig:node0_ui}に示すようにAR展品がシーン内に初期化され、インタラクションが開始される。

% Node0のUIの図
\begin{figure}[htbp]
\centering
\includegraphics[width=80mm]{./figs/提案システム/Node/Node0UI.png}
\caption[Node0UI]{Node0UI [5].}
\label{fig:node0_ui}
\end{figure}

\subsubsection{AOTストリッピングの防止 (AOT Stripping Prevention)}

UnityのIL2CPPビルドでは、ビルド時に使用されていないコードや型情報はファイルサイズ削減のために削除(ストリッピング)される。しかし、動的にロードされるNode側のスクリプトが、アプリ本体側で削除された型に依存している場合、実行時エラーが発生する。

これを防ぐため、本システムでは link.xml ファイルを定義し(図\ref{fig:linkxml_config})、Nodeで使用する可能性のあるアセンブリや型を明示的に記述することで、ビルド時のストリッピングを回避している。これにより、ARShowNodeで開発された任意のスクリプトが、ARShowアプリ上で正しく動作することを保証している。

% Link.xmlコードの配置図
\begin{figure}[htbp]
\centering
\includegraphics[width=80mm]{./figs/提案システム/ARShow/Linkxml配置.png}
\caption[Linkxml配置]{Linkxml配置 [5].}
\label{fig:linkxml_config}
\end{figure}

\subsection{サーバ}

AssetBundleのホスティングには、開発PC機上で動作するASP.NET Coreベースの静的ファイルサーバを用いた。本サーバはMinimal API構成で実装されており、HTTPリクエストに応じてAssetBundleファイルを提供する単純かつ高速な仕様となっている。

「ARShow」プロジェクトのメニューバーには、図\ref{fig:server_tool}に示すように、このサーバの「起動」「停止」「ディレクトリ清掃」を制御する機能も統合されており、展示運用時のサーバ管理を容易にしている。なお、開発環境(ローカルLAN)においては、ARShowアプリにサーバからAssetBundleをダウンロードする用のIPアドレスを開発PC機の実際アドレースと一致させる必要がある。

% 提案されたServerToolの図
\begin{figure}[htbp]
\centering
\includegraphics[width=80mm]{./figs/提案システム/ARShow/ServerTool.png}
\caption[ServerTool]{ServerTool [5].}
\label{fig:server_tool}
\end{figure}

\section{UIとインタラクション}

本節では、本システムにおいて実装された3つ代表的な展示作品(Node)を例に挙げ、鑑賞者が体験するユーザーインターフェース(UI)およびインタラクションの詳細について述べる。各Nodeはそれぞれ異なるメディア形式(複合UI、映像、3Dモデル)を扱っており、システムの汎用性を示している。

\subsection{Node0: 複合的なAR展示インターフェース}

Node0は、本システムの中で最も機能的に複雑な展示例であり、文化財の3Dモデル表示と解説テキスト、および音声操作を組み合わせた複合的なARコンテンツである。実際のUI表示を図\ref{fig:node0_ui}に示す。

% Node0のUIの図
\begin{figure}[htbp]
\centering
\includegraphics[width=80mm]{./figs/提案システム/Node/Node0UI.png}
\caption[Node0UI]{Node0UI [5].}
\label{fig:node0_ui}
\end{figure}

\subsubsection{インターフェース構成}

本NodeのUIは、Meta Horizon OS UI Setをベースに構築されており、Meta Questの標準的なシステムUIと親和性の高いデザインを採用している。画面構成は主に「音声のテキスト化フィードバック」と「操作用テキストキャンバス」の2つのモジュールから成る。テキストキャンバスは、機能に応じて以下の3つのセクションに区分されている。

\begin{itemize}
    \item \textbf{スクリーンリーダー制御}: 図\ref{fig:node0_reader}に示すように、解説音声の「再生(Play)」「一時停止(Pause)」「停止(Stop)」を行うボタン群が配置されている。
    \item \textbf{多国言語制御}: 文化財の解説テキストを表示するエリアおよび言語選択ドロップダウンメニューである。言語切り替えにより、テキストと読み上げ音声が即座に変更される。
    \item \textbf{音声コマンド制御}: 「Listen」ボタンと認識結果を表示するテキストボックスから構成される(図\ref{fig:node0_listen})。
\end{itemize}

\begin{figure}[htbp]
\centering
% Node0ReaderボタンUIの図
\begin{minipage}{0.48\textwidth}
  \centering
  \includegraphics[width=\linewidth]{./figs/提案システム/Node/Node0UIReader.png}
  \caption[Node0UIReader]{Node0UIReader [5].}
  \label{fig:node0_reader}
\end{minipage}
\hfill
% Node0ListenボタンUIの図
\begin{minipage}{0.48\textwidth}
  \centering
  \includegraphics[width=\linewidth]{./figs/提案システム/Node/Node0UIListen.png}
  \caption[Node0UIListen]{Node0UIListen [5].}
  \label{fig:node0_listen}
\end{minipage}
\end{figure}

\subsubsection{インタラクション}

鑑賞者は、ハンドトラッキング機能を用いた以下のように直感的な操作が可能である。

\begin{itemize}
    \item \textbf{タッチ操作(Poke)}: 仮想ボタンを指先で直接押すことで、再生制御やモード切替を行う(図\ref{fig:node0_poke})。
    \item \textbf{遠隔操作(Ray)}: 手から発せられるレイ(光線)を用いて、離れた位置にあるUI要素を選択と操作する(図\ref{fig:node0_ray})。
    \item \textbf{把持操作(Grab)}: UIパネルや3Dモデルを「掴む」ジェスチャにより、鑑賞しやすい位置や角度へ自由に移動させることができる(図\ref{fig:node0_grab})。
\end{itemize}

\begin{figure}[htbp]
\centering
% Node0ボタンUIのPoke図
\begin{minipage}{0.32\textwidth}
  \centering
  \includegraphics[width=\linewidth]{./figs/提案システム/Node/Node0UIPoke.png}
  \caption[Node0UIPoke]{Node0UIPoke [5].}
  \label{fig:node0_poke}
\end{minipage}
\hfill
% Node0ボタンUIのRay図
\begin{minipage}{0.32\textwidth}
  \centering
  \includegraphics[width=\linewidth]{./figs/提案システム/Node/Node0UIRay.png}
  \caption[Node0UIRay]{Node0UIRay [5].}
  \label{fig:node0_ray}
\end{minipage}
\hfill
% Node0ボタンUIのGrab図
\begin{minipage}{0.32\textwidth}
  \centering
  \includegraphics[width=\linewidth]{./figs/提案システム/Node/Node0UIGrab.png}
  \caption[Node0UIGrab]{Node0UIGrab [5].}
  \label{fig:node0_grab}
\end{minipage}
\end{figure}

これらのインタラクションを実現するため、技術的には PointableCanvasModule システムを利用している(図\ref{fig:pointable_canvas})。通常、このイベントシステムはUnityシーン内に常駐する必要があるが、本システムでは動的にロードされるNode側(ARShowNode)で定義されたイベント設定を、実行時に本体アプリ(ARShow)のシーンへ正しく引き継ぐ仕組みを実装することで、スムーズな操作性を確保している。

% PointableCanvasModuleの配置図
\begin{figure}[htbp]
\centering
\includegraphics[width=80mm]{./figs/提案システム/ARShow/PointableCanvasModule.png}
\caption[PointableCanvasModule]{PointableCanvasModule [5].}
\label{fig:pointable_canvas}
\end{figure}

\subsubsection{音声コマンド制御}

「Listen」ボタンを押下すると音声認識モードへ移行し、Wit.aiを介したボイスコマンドによる操作が可能となる。図\ref{fig:voice_ignore}と図\ref{fig:voice_status}に示すように、システムは待機状態(Ignore)から聞き取り状態(Listen)へと遷移する。

\begin{figure}[htbp]
\centering
% Node0のVoiceIgnoreStatus図
\begin{minipage}{0.48\textwidth}
  \centering
  \includegraphics[width=\linewidth]{./figs/提案システム/Node/Node0VoiceIgnoreStatus.png}
  \caption[Node0VoiceIgnoreStatus]{Node0VoiceIgnoreStatus [5].}
  \label{fig:voice_ignore}
\end{minipage}
\hfill
% Node0のVoiceListenStatus図
\begin{minipage}{0.48\textwidth}
  \centering
  \includegraphics[width=\linewidth]{./figs/提案システム/Node/Node0VoiceListenStatus.png}
  \caption[Node0VoiceListenStatus]{Node0VoiceListenStatus [5].}
  \label{fig:voice_status}
\end{minipage}
\end{figure}

例えば、所定のキーワードを発話することで、ボタン操作なしに解説の再生や言語変更を行うことができる。認識された発話内容は左側UIモジュールにテキストとしてフィードバックされ、確実な入力を支援する。

図\ref{fig:voice_rec_ch}と図\ref{fig:voice_rec_en}に中国語および英語の言語に切り替えす結果を示す。

\begin{figure}[htbp]
\centering
% Node0のVoiceNviChの図
\begin{minipage}{0.48\textwidth}
  \centering
  \includegraphics[width=\linewidth]{./figs/提案システム/Node/Node0VoiceNviCh.png}
  \caption[Node0VoiceNviCh]{Node0VoiceNviCh [5].}
  \label{fig:voice_rec_ch}
\end{minipage}
\hfill
% Node0のVoiceNviEnの図
\begin{minipage}{0.48\textwidth}
  \centering
  \includegraphics[width=\linewidth]{./figs/提案システム/Node/Node0VoiceNviEn.png}
  \caption[Node0VoiceNviEn]{Node0VoiceNviEn [5].}
  \label{fig:voice_rec_en}
\end{minipage}
\end{figure}

図\ref{fig:voice_start}と図\ref{fig:voice_stop}に音声コマンドによる再生開始と停止の挙動を示す。

\begin{figure}[htbp]
\centering
% Node0のVoicePlayStart図
\begin{minipage}{0.48\textwidth}
  \centering
  \includegraphics[width=\linewidth]{./figs/提案システム/Node/Node0VoicePlayStart.png}
  \caption[Node0VoicePlayStart]{Node0VoicePlayStart [5].}
  \label{fig:voice_start}
\end{minipage}
\hfill
% Node0のVoicePlayStop図
\begin{minipage}{0.48\textwidth}
  \centering
  \includegraphics[width=\linewidth]{./figs/提案システム/Node/Node0VoicePlayStop.png}
  \caption[Node0VoicePlayStop]{Node0VoicePlayStop [5].}
  \label{fig:voice_stop}
\end{minipage}
\end{figure}

\subsection{Node1: 映像コンテンツの空間配置}

Node1は、映像メディアをAR空間内に配置する展示例である(図\ref{fig:node1_ui})。Unityのプリミティブ形状であるQuad(板ポリゴン)にVideo Playerコンポーネントを付加し、独自の録画映像と音声を再生する構成となっている。本NodeにもGrabインタラクションが付与されており、鑑賞者は空中に浮かぶスクリーンを手に取り、壁面に配置したり、目の前に引き寄せて細部を確認したりといった、物理的なスクリーンと同様の取り回しが可能である。これは、動画解説パネルとしての利用を想定した実装である。

% Node1のUIの図
\begin{figure}[htbp]
\centering
\includegraphics[width=80mm]{./figs/提案システム/Node/Node1UI.png}
\caption[Node1UI]{Node1UI [5].}
\label{fig:node1_ui}
\end{figure}

\subsection{Node2: 3Dモデル(文化財)の展示}

Node2は、静的な3Dオブジェクトの展示に特化した最小構成の例である(図\ref{fig:node2_ui})。ここでは、Sketchfabより取得した青銅器の3Dモデル(glTF形式)をPrefab化し、AssetBundleとして配信している。Node1同様、Grabインタラクションが設定されており、鑑賞者は貴重な文化財モデルを仮想的に手に取り、あらゆる角度から詳細に観察することができる。このNodeは、複雑なスクリプトを含まない純粋なアセットデータも、本システムを通じて問題なく配信や操作可能であることを実証している。

\clearpage
% Node2のUIの図
\begin{figure}[htbp]
\centering
\includegraphics[width=80mm]{./figs/提案システム/Node/Node2UI.png}
\caption[Node2UI]{Node2UI [5].}
\label{fig:node2_ui}
\end{figure}

\section{ワークフロー}

本節では、本システムを用いたAR展示の制作から鑑賞に至るまでの具体的なワークフローについて、制作サイド(キュレーター)と鑑賞サイド(閲覧者)の双方の視点から述べる。

\subsection{制作サイド(キュレーター)}

キュレーター側の作業は、Unityプロジェクト「ARShowNode」を用いて行われる。制作から公開に至る全体のワークフローを図\ref{fig:creator_workflow}に示す。

% 制作者ワークフロー図
\begin{figure}[htbp]
\centering
\includegraphics[width=80mm]{./figs/提案システム/制作者ワークフロー.png}
\caption[制作者ワークフロー]{制作者ワークフロー [5].}
\label{fig:creator_workflow}
\end{figure}

\subsubsection{ARShowNodeプロジェクトの配置}

まず、キュレーターは提供される「ARShowNode」プロジェクトを開発環境に展開する。このプロジェクトには、Meta XR All-in-One SDK、HybridCLR、および本研究で開発した専用ツールキット等の依存ライブラリが事前設定されている。キュレーターは、Unityエディタ上でコンパイルエラーがない状態を確認し、自身のコンテンツ(Prefabやスクリプト)の制作を開始する。

\subsubsection{Assemblyのビルド}

作品のロジック(C\#スクリプト)をホットアップデート可能な形式に変換するため、HybridCLRのビルド機能を実行する。まず、HybridCLRメニューから Install を実行し、環境を初期化する。次にCompileDllコマンドを実行し、ターゲットプラットフォーム(Android)向けのDLL(AOTメタデータおよびホットアップデート用アセンブリ)を出力する。そしてGenerateコマンドを実行し、ブリッジコード等を生成する。

\subsubsection{AssetBundleの事前準備}

DLLのビルド完了後、上述の本研究が提供したUnityツール(ARShowTool)を用いて、以下の手順で配信準備を行う。

\begin{enumerate}
    \item \textbf{DLLの配置}: ツールメニューの「Move DLLs」を実行し、生成されたDLLファイルをUnityプロジェクトのAssetsフォルダへ複製する。
    \item \textbf{バンドル設定}: 各作品のリソース(Prefab, DLL等)に対し、一意のAssetBundleラベル(例: node0)を付与する。
    \item \textbf{ビルド実行}: ツールメニューの「Build AssetBundles」を実行する。これにより、ラベル付けされたリソースが一つのAssetBundleファイルとしてパッケージ化される。
    \item \textbf{QRコード生成}: ツールメニューの「Generate QR」を実行し、各AssetBundleに対応するQRコード画像を生成する。
\end{enumerate}

\subsubsection{サーバへのアップロード}

最後に、ツールメニューの「Upload to Server」を実行する。これにより、生成された全てのAssetBundleファイルが、LAN内で稼働している静的ファイルサーバの公開ディレクトリへ自動的に転送される。以上で、展示コンテンツの公開作業は完了である。

\subsection{鑑賞サイド(鑑賞者)}

鑑賞者は、HMD(Meta Quest 3)を装着し、図\ref{fig:viewer_workflow}に示す手順で展示を体験する。

\clearpage
% 閲覧者ワークフロー図
\begin{figure}[htbp]
\centering
\includegraphics[width=80mm]{./figs/提案システム/閲覧者ワークフロー.png}
\caption[閲覧者ワークフロー]{閲覧者ワークフロー [5].}
\label{fig:viewer_workflow}
\end{figure}

\subsubsection{ARShow AR APPのインストール}

事前に、Meta Quest Developer Hub (MQDH) 等を経由して、閲覧用アプリ「ARShow」をデバイスにインストールする。

\subsubsection{スキャンモードによる開始}

アプリを起動すると、空間上に「ScanQr」ボタンが表示される。これをクリックすると、パススルーカメラを用いたQRコードスキャンモードに移行する。鑑賞者が展示会場に掲示されたQRコードに視線を向けると、システムは自動的にコードを認識して解析する。

\subsubsection{プログラムの実行フロー}

QRコードの認識後、システムは以下のフローを自動的に実行する。

\begin{enumerate}
    \item \textbf{ダウンロード}: 解析されたIDに基づき、サーバから対応するAssetBundleをダウンロードする。進捗状況は空間上のプログレスバーで可視化される。
    \item \textbf{アセンブリロード}: ダウンロード完了後、AssetBundle内のAOTメタデータDLLとホットアップデートDLLをメモリに展開する。
    \item \textbf{初期化(Entry Point)}: ロードされたアセンブリ内の Entry.cs を特定し、その初期化メソッドを実行する。この段階で、展示作品のPrefabがシーン内に生成(Instantiate)され、必要なコンポーネントが動的にアタッチされる。
    \item \textbf{登録と管理}: 生成された作品はIDと共に内部辞書に登録される。スキャンモードは終了し、鑑賞者は作品とのインタラクションが可能となる。
\end{enumerate}

なお、既にロード済みのQRコードを再スキャンした場合は、重複ロードを防ぐためコンソールに警告が表示され、およびスキャンモードが解除される仕様となっている。


\chapter{評価実験と考察}

本章では、前章で提案した「ARShowNode」および「ARShow」から成るAR展示プラットフォームの有用性と実用性を検証するために実施した被験者実験について述べる。本実験では、システムユーザビリティの観点から定量的な評価を行うとともに、実際の展示運用を模したワークフローを通じて、提案システムが抱える課題と可能性について考察を行う。

\section{実験仮説}

本研究の目的は、HMDを用いたAR展示において、制作者(キュレーター)のコンテンツ更新負荷を軽減し、かつ閲覧者(鑑賞者)に対して直感的で没入感のある鑑賞体験を提供することにある。この目的に基づき、本実験では以下の3つの仮説を立て、その検証を行うこととした。

\subsection{仮説1}

制作者における制作効率の向上: 提案システムが提供する「ARShowTool」およびホットアップデート技術を用いることで、制作者はアプリケーション本体の再ビルドを行うことなく、短時間かつ低コストで複雑なロジックを含むARコンテンツの配信が可能である。これにより、従来のアプリ配布型展示と比較して運用コストが著しく低減される。

\subsection{仮説2}

閲覧者におけるアクセシビリティの向上: QRコードを物理的なインターフェースとして用いる「ARShow」アプリの操作体系は、HMD特有のコントローラ操作やUIに不慣れなユーザーに対しても、物理空間と情報空間をシームレスに接続する直感的なメタファーとして機能する。これにより、複雑なメニュー操作や事前の導入手順を学習することなく、即座にAR体験を開始できる高いアクセシビリティを有する。

\subsection{仮説3}

システムの総合的なユーザビリティの向上: 制作者と閲覧者の双方が、それぞれの役割においてシステムを円滑に操作でき、コンテンツの制作から配信、そして鑑賞に至る一連のプロセス(エコシステム)が、技術的な破綻や遅延なく統合的に機能する。これにより、本システムが単なる実験的なプロトタイプに留まらず、実用的な展示プラットフォームとしての持続可能性と堅牢性を有していることが実証される。

\section{アンケート}

本実験における定量的評価指標として、システムユーザビリティスケール(System Usability Scale、以下SUS)を採用した。SUSは、Brooke(1996)によって考案されたユーザビリティ評価のための標準的なアンケート手法であり、システムの有効性、効率性、満足度を包括的に測定することが可能である。

本研究においてSUSを選択した理由は、以下の2点である。第一に、SUSは技術的なシステムから日常的な製品まで幅広い対象に適用可能であり、信頼性と妥当性が広く認められている点である。第二に、わずか10項目の質問で構成されており、被験者への負担を最小限に抑えつつ、比較可能なスコア(0点から100点)を算出できる点である。一般的に、SUSスコアが68点以上であれば平均以上のユーザビリティを有していると判断される。

質問項目は5段階のリッカート尺度(1: 全くそう思わない 〜 5: 強くそう思う)で回答を求めた。なお、質問文中の「システム」という単語は、被験者の役割に応じて「AR制作ツール(ARShowNode)」または「AR閲覧アプリ(ARShow)」と読み替えるよう教示を行った。

[表3: 実験で使用したSUS質問項目一覧] (ここにSUSの10項目の質問内容、および回答用スケールの例を示す表を挿入)

\section{被験者の募集}

本実験の被験者は、Googleフォームを用いた公募により選定した。実験の性質上、システムの「制作側」と「閲覧側」双方の視点が必要となるため、計4名の被験者を採用し、それぞれ以下の役割を割り当てた。

\subsection{制作者}

制作者(Creator)役:2名

選定条件:Unityエンジンの基本的な操作(エディタ操作、C\#スクリプトの理解)に習熟していること。

理由:提案システム「ARShowNode」はUnity開発者を対象としたツールキットであるため、一定の開発スキルを有するユーザーによる評価が不可欠であるため。

\subsection{閲覧者}

閲覧者(Viewer)役:2名

選定条件:特段の技術的背景は問わないが、VR/ARデバイスの使用に抵抗がないこと。

理由:一般の美術館や展示会への来場者を想定し、専門知識を持たないユーザーでも直感的に操作可能かを検証するため。

このように役割を分担することで、実際の展示会における「キュレーター(作品提供者)」と「オーディエンス(鑑賞者)」の関係性を模倣し、システム全体のエコシステムとしての妥当性を検証することを意図した。

[表4: 被験者の属性と経験年数] (被験者A〜Dの年齢、性別、Unity使用歴、VR/AR体験頻度などをまとめた属性表を挿入)

\section{提案システム操作方式の紹介}

\subsection{制作者(Creator)への操作説明}

制作者には、Unityプロジェクト「ARShowNode」および本研究で実装した専用エディタ拡張「ARShowTool」の使用方法を説明した。制作者は、Unityエディタ上で3Dモデルの配置やC\#スクリプトの記述(HybridCLR対応)を行い、展品(Node)としての体裁を整えた後、以下の6つの手順を順次実行することでコンテンツの配信が完了することを教授した。

1、DLLのコンパイル (CompileDll): HybridCLRのメニューからCompileDllを実行し、アクティブなビルドターゲット(Android/Quest)に対応したホットアップデート用DLLを出力する。

2、コード生成 (Generate All): 続いてHybridCLRメニューからGenerate(All)を実行し、AOTメタデータおよびC\#とC++間のブリッジコードを生成する。これにより、アプリ本体を更新することなくロジックを配信可能な状態にする。

3、アセンブリの配置 (Move DLLs): ARShowToolメニューのMove DLLsを実行し、生成されたDLLファイルをUnityプロジェクト内のAssetsディレクトリへ自動的に複製する。

4、アセットバンドルの構築 (Build AssetBundles): ARShowToolメニューのBuild AssetBundlesを実行する。事前に設定されたラベル(例: node0, node1)に基づき、Prefab、リソース、および手順3で配置したDLLを含んだAssetBundleファイルを生成する。

5、QRコードの生成 (Generate QR): ARShowToolメニューのGenerate QRを実行し、各AssetBundleの識別子情報を格納したQRコード画像を生成する。制作者はこの画像を保存し、閲覧者に提示する準備を行う。

6、サーバへの配信 (Upload to Server): 最後にARShowToolメニューのUpload to Serverを実行する。生成されたAssetBundle群が、稼働中の静的ファイルサーバの公開ディレクトリへ一括転送され、配信可能な状態となる。

\subsection{閲覧者(Viewer)への操作説明}

閲覧者には、Meta Quest 3にインストールされた「ARShow」アプリの操作方法を説明した。アプリはHMDのパススルー機能を用いたARモードで動作し、以下の手順で鑑賞を行う旨を伝えた。

1、スキャンモードの入り: アプリ起動後、空間上に表示されるUIパネルの「ScanQr」ボタンを指で押下(Poke操作)し、QRコード認識待機モードへ移行する。

2、QRコードの検出: パススルー映像越しに、制作者から提示されたQRコードに視線を向ける。ZXingライブラリによりコードが認識されると、自動的にスキャンモードが終了し、コンテンツのロード処理が開始される。

3、ダウンロードと展開: サーバからAssetBundleのダウンロードが開始され、空間上のプログレスバーに進捗が表示される。完了後、HybridCLRによるDLLのロードとEntryポイントの実行が自動的に行われ、目の前に展品が出現する。

4、インタラクション: 出現した展品に対し、ハンドトラッキングを用いた直感的な操作が可能であることを説明した。具体的には、オブジェクトを直接掴んで移動させる「Grab」、遠くの対象を指し示す「Ray」、ボタンを押す「Poke」、および音声コマンドによる操作を実演と共にレクチャーした。

\section{実験の流れ}

本実験は、実験室環境にて実施した。ハードウェアリソースの制約(開発用PC 1台、Meta Quest 3 1台)を考慮し、被験者を2つのグループに分け、以下の手順で順次実験を行った。

なお、実験の複雑度を制御し、純粋に「提案ツールの操作性」と「鑑賞体験」を評価するため、制作者は新規プロジェクトの立ち上げを行うのではなく、あらかじめ必要な設定(SDK導入、HybridCLR設定済み)が完了している「ARShowNode」プロジェクトを使用した。また、コンテンツ配信用の静的ファイルサーバは実験者が事前に起動し、制作者がサーバ管理を行う必要はないものとした。また、各制作者の実操作前に、前回の実験データによる干渉を防ぐため、ARShowNodeプロジェクト内の生成物(AssetBundles、QRコード画像)およびサーバ上のファイルを完全に削除し、初期状態へのリセットを行った。

\subsection{実験プロトコル}

\subsection{グループ1(制作者1 + 閲覧者1・2)}

制作フェーズ: 制作者1は、ARShowToolを用いて2つの異なる展品「Node0(解説付き文化財)」および「Node1(映像展示)」のビルドからアップロードまでを行う。生成された2つのQRコードを印刷または画面表示により閲覧者に提供する。

閲覧フェーズ: 閲覧者1がQuest 3を装着し、Node0、Node1の順にQRコードをスキャンして体験を行う。体験終了後、デバイスを閲覧者2に交代し、同様にNode0、Node1の体験を行う。

\subsection{グループ2(制作者2 + 閲覧者1・2)}

制作フェーズ: 制作者2は、同様の手順で「Node0」および「Node2(静的3Dモデル)」のビルドとアップロードを行う。生成されたQRコードを閲覧者に提供する。

閲覧フェーズ: グループ1と同様に、閲覧者1、閲覧者2の順でQuest 3を装着し、Node0およびNode2の体験を行う。

全ての体験が終了した後、4名の被験者はGoogleフォームにてSUSアンケートおよび自由記述によるフィードバックへの回答を行った。

[図8: 実験フローチャート] (事前準備からグループ分け、制作、閲覧、アンケート回答に至る一連の流れを示すフロー図を挿入)

\section{実験結果}

4名の被験者から得られたSUSアンケートの回答を集計し、SUSスコア算出定義に基づき0点から100点のスコアに換算した。

結果の概要は以下の通りである。全体の平均スコアは XX.X点 であり、一般的な平均基準とされる68点を大きく上回る結果となった。役割別に見ると、閲覧者(Viewer)の平均スコアは XX.X点 と極めて高く、制作者(Creator)の平均スコアは XX.X点 であった。

自由記述によるフィードバックにおいては、各役割の体験を反映した具体的な意見が得られた。 閲覧者からは、「QRコードを見るだけで体験が始まる点が非常にスムーズだった」「HMDをつけたまま複雑なメニュー操作をしなくて済むのが良い」といった、操作の直感性と身体的負荷の少なさを評価する肯定的な意見が多く寄せられた。 一方、制作者からは、「ツールを使えばワンクリックでアップロードまで完了するのは非常に便利である」というワークフローの効率性を評価する声があった一方で、「エラーが出た際のデバッグが難しい」「実機とエディタの挙動の違いに戸惑った」といった、ホットアップデート開発プロセス特有の課題も指摘された。

[表5: SUSスコア集計結果] (被験者ごとの内訳、役割別平均、全体平均、標準偏差を示す表を挿入)

[図9: 質問項目別平均スコア] (SUSの10項目ごとの平均点を棒グラフで示し、どの項目が高評価/低評価であったかを可視化する図を挿入)

\section{考察}

本節では、アンケート結果および実験中の観察に基づき、提案システムの評価について考察する。

制作者によるSUSスコアが基準点を上回ったこと、および「ワンクリックでの配信」に対する肯定的評価は仮説1を支持する結果である。特に、Unityエディタ上で完結するツールキット(ARShowTool)の提供により、スクリプトのコンパイルからアセットバンドルの生成、サーバ転送までの一連の作業が自動化された点は、従来のアプリ本体の再ビルドを要する手法と比較して、運用コストと時間の著しい低減を実現しているといえる。しかし、スコアが閲覧者に比べてやや低かった要因として、HybridCLRを用いた開発特有の制約(AOTメタデータの管理や、実機でのみ発生するランタイムエラーへの対処など)が、従来のUnity開発フローとは異なる学習コストを要したためと考えられる。したがって、仮説1の有効性は示されたものの、実用化に向けてはツールのUI改善やエラーログの可視化機能の拡充など、開発者体験(DX)の向上が今後の課題である。

閲覧者による極めて高いSUSスコアと、「スムーズな開始」に関するフィードバックは仮説2を支持している。本システムが採用したQRコードをトリガーとする手法は、HMD特有のコントローラ操作やUIに不慣れなユーザーに対しても、物理空間と情報空間をシームレスに接続する直感的なメタファーとして機能したことが確認された。これにより、ユーザーは複雑な導入手順を学習することなく、即座にAR体験へ没入することが可能となり、本システムが高いアクセシビリティを有していることが実証された。

制作者1から制作者2へと配信者が交代し、サーバ上のコンテンツが完全に入れ替わった際も、閲覧者側のアプリは再起動することなく、新しいQRコードを読み込むだけで即座に最新のコンテンツに対応できた。また、LAN環境下での動的ロードにおいても遅延やクラッシュ等の重大な不具合は発生しなかった。これらの事実は、制作者と閲覧者を繋ぐエコシステムが技術的な破綻なく統合的に機能していることを示しており、仮説3を裏付けるものである。提案システムは、展示内容の頻繁な更新や複数作家による共同展示といった実際の運用シナリオにおいても、十分な堅牢性と再現性を備えた実用的なプラットフォームであると結論付けられる。

以上の検証より、本研究で提案した「ARShowNode」および「ARShow」システムは、AR展示における制作者と鑑賞者の双方に対し、従来のアプリ配布型モデルが抱えていた課題を解決する有効なソリューションであることが確認された。今後は、被験者数を増やした大規模な実験や、インターネット経由での遠隔配信実験を行い、さらなる検証とシステムの洗練を進める必要がある。

\chapter{まとめ}

まとめと今後の展望を書く
\chapter*{謝辞}
\quad ここに謝辞を書く

\rightline{yyyy年3月}
\rightline{姓 名}


% 参考文献
\begin{thebibliography}{99}

\bibitem{ref:journal}
Ohlei, A., Schumacher, T., \& Herczeg, M. (2020). An Augmented Reality Tour Creator for Museums with Dynamic Asset Collections. In Augmented Reality, Virtual Reality, and Computer Graphics (LNCS 12243, pp. 15-31). Springer. 
\bibitem{ref:journal}
Duanmu, Q., Dai, T., Cai, Y., \& Herman, J. (2023). AR MUSE: Designing and Implementing a Solution for Accessible Augmented Reality Exhibition. Worcester Polytechnic Institute. 
\bibitem{ref:journal}
Bekele, M.K. Clouds-Based Collaborative and Multi-Modal Mixed Reality for Virtual Heritage. Heritage 2021, 4, 1447-1459. 
\bibitem{ref:journal}
Kidman, B. A Platform for in-Situ Creation of Markerless, Location-Based Augmented Reality Content. Master's Thesis, Dartmouth College, 2023. 
\bibitem{ref:journal}
飛田博章, 渡辺光太郎, 山川美咲, 小島瑛里子. 拡張現実を用いた作品に対するコメントを共有することによる対話型美術鑑賞の支援. 日本・美術による学び学会誌, 第 6 巻, 第 1 号.
\bibitem{ref:journal}
Leandro Soares Guedes, Luiz André Marques, and Gabriellen Vitório. "Enhancing interaction and accessibility in museums and exhibitions with Augmented Reality and Screen Readers." Università della Svizzera italiana / Federal Institute of Mato Grosso do Sul. 
\bibitem{ref:journal}
伏田昌弘, 赤羽亨. "画像マーカーベースの AR を用いた音声ガイドの試作." 情報処理学会 インタラクション 2024, IA-16, pp. 253-256, 2024. 
\bibitem{ref:journal}
Kyriakou, P. and Hermon, S.: Can I touch this? Using Natural Interaction in a Museum Augmented Reality System, Digital Applications in Archaeology and Cultural Heritage, Vol. 12, e00088 (2018). 
\bibitem{ref:journal}
Liu, Y., Spierling, U., Rau, L. and Dörner, R.: Handheld vs. Head-Mounted AR Interaction Patterns for Museums or Guided Tours, Intelligent Technologies for Interactive Entertainment (INTETAIN 2020), LNICST 377, pp. 229-242 (2021). 
\bibitem{ref:journal}
Liu, Y., Bitter, J.L. and Spierling, U.: Evaluating Interaction Challenges of Head-Mounted Device-based Augmented Reality Applications for First-time Users at Museums and Exhibitions. 
\bibitem{ref:journal}
Ramy Hammady, Minhua Ma, Ziad AL-Kalha, and Carl Strathearn. A framework for constructing and evaluating the role of MR as a holographic virtual guide in museums. Virtual Reality, Vol. 25, pp. 1-25, 2021. 
\bibitem{ref:journal}
井上道哉, 長澤可也. 綾瀬市埋蔵文化財の VR 、 AR コンテンツ化による地域活性化 湘南工科大学紀要, Vol. 55, No. 1, pp. 41-47, 2021. 
\bibitem{ref:journal}
Weiting Hou: Augmented Reality Museum Visiting Application based on the Microsoft HoloLens, Journal of Physics: Conference Series, Vol. 1237, 052018, 2019. 
\bibitem{ref:journal}
赤嶺有平: 拡張現実を用いた博物館における双方向メディア型ガイダンスシステムの開発. 科学研究費助成事業 研究成果報告書, 課題番号 19K13045, 2021. 
\bibitem{ref:journal}
星野浩司: AR 型遠隔学習支援システム「 AI Aquarium 」の開発. 九州産業大学芸術学部研究報告, 第 56 巻, pp. 38-41, 2024. 
\bibitem{ref:journal}
Robert Lee Seligmann. "Web-based Client for Remote Rendered Virtual Reality". Master's Thesis, Aalto University, 2020. 
\bibitem{ref:journal}
Viktor Kelkkanen, Markus Fiedler, and David Lindero. "Synchronous Remote Rendering for VR". International Journal of Computer Games Technology, Vol. 2021, Article ID 6676644, 2021. 
\bibitem{ref:journal}
Noor Hammad, Thomas Eiszler, Robert Gazda, John Cartmell, Erik Harpstead, and Jessica Hammer. "V-Light: Leveraging Edge Computing For The Design of Mobile Augmented Reality Games". In Proceedings of the 18th International Conference on the Foundations of Digital Games (FDG '23), 2023.


% =====================
% 参考文献 書き方例

% \bibitem{ref-related-work}
% 論文の読み方・書き方, 金森 由博, \url{http://kanamori.cs.tsukuba.ac.jp/docs/how_to_read_and_write_papers.html}, 2021/09/28.
% \bibitem{ref-jsps} 
% 研究者のみなさまへ~責任ある研究活動を目指して~, 国立研究開発法人科学技術振興機構, \url{https://www.jst.go.jp/researchintegrity/shiryo/pamph_for_researcher.pdf}, 2020. 

% \bibitem{ref-ieice} 
% 和文論文誌投稿のしおり, 電子情報通信学会, \url{https://www.ieice.org/jpn/shiori/iss_2.html#2.6}, 2021/09/28.

% \bibitem{ref-ipsj} 
% 論文誌ジャーナル原稿執筆案内, 情報処理学会, \url{https://www.ipsj.or.jp/journal/submit/ronbun_j_prms.html}, 2021/09/28.

% \bibitem{ref:journal}
% Almuzaini, K.K., Joshi, S., Ojo, S. et al.,
%    "Optimization of the operational state's routing for mobile wireless sensor networks," 
%    Wireless Networks, pp. 1-15, 2023.

% \bibitem{ref-journal} 
% 著者名, ``論文タイトル,'' 雑誌名, vol, no, page, year.

% \bibitem{ref-journal-ex} 
% 國田 樹, 遠藤聡志, ``学術論文の出典記載例,'' 知能情報学会誌, vol. 3, no. 2, pp.8-13, 2021.

% \bibitem{ref:book}
% 中垣俊之, ``粘菌その驚くべき知性,'' 株式会社PHP 研究所, 東京, 2010.

% \bibitem{ref-book}
% 著者名, ``書籍タイトル,'' (編集者名), 出版社名, 発行都市名, 発行年.

% \bibitem{ref-book-ex}
% 國田樹, ``著書の出典記載例,'' 知能情報出版, 沖縄, 2021.

% \bibitem{ref-proceedings}
% 著者名, ``論文タイトル,'' 学会名もしくは会議名, no.論文番号, ページ, 開催都市名, 開催国名, year. 

% \bibitem{ref-proceedings-ex}
% 國田樹, 遠藤聡志, ``学会論文の出典記載例'' 第2回知能情報国際会議, no.2-1234, pp.1-8, Okinawa, Japan, 2021. 

% \bibitem{ref:web}
% What is cloud cost optimization?, IBM, 
% \url{https://www.ibm.com/blog/what-is-cloud-cost-optimization/}, 2024/01/11.

% \bibitem{ref-web}
% 著者名(サイト管理者と同一の場合は省略可), Webページタイトル, サイト管理者名等, URL(url命令を使用すること), 参照年月日.  

% \bibitem{ref-report1}
% 見延庄太郎,理系のためのレポート・論文完全ナビ,講談社, 2016.

% \bibitem{ref-report2}
% 福地健太郎,理工系のためのよい文章の書き方,翔泳社, 2019.

\end{thebibliography}


\end{document}

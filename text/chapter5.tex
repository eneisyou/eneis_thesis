\chapter{評価実験と考察}

本章では、前章で提案した「ARShowNode」および「ARShow」から成るAR展示プラットフォームの有用性と実用性を検証するために実施した被験者実験について述べる。本実験では、システムユーザビリティの観点から定量的な評価を行うとともに、実際の展示運用を模したワークフローを通じて、提案システムが抱える課題と可能性について考察を行う。

\section{実験仮説}

本研究の目的は、HMDを用いたAR展示において、制作者(キュレーター)のコンテンツ更新負荷を軽減し、かつ閲覧者(鑑賞者)に対して直感的で没入感のある鑑賞体験を提供することにある。この目的に基づき、本実験では以下の3つの仮説を立て、その検証を行うこととした。

\subsection{仮説1}

制作者における制作効率の向上: 提案システムが提供する「ARShowTool」およびホットアップデート技術を用いることで、制作者はアプリケーション本体の再ビルドを行うことなく、短時間かつ低コストで複雑なロジックを含むARコンテンツの配信が可能である。これにより、従来のアプリ配布型展示と比較して運用コストが著しく低減される。

\subsection{仮説2}

閲覧者におけるアクセシビリティの向上: QRコードを物理的なインターフェースとして用いる「ARShow」アプリの操作体系は、HMD特有のコントローラ操作やUIに不慣れなユーザーに対しても、物理空間と情報空間をシームレスに接続する直感的なメタファーとして機能する。これにより、複雑なメニュー操作や事前の導入手順を学習することなく、即座にAR体験を開始できる高いアクセシビリティを有する。

\subsection{仮説3}

システムの総合的なユーザビリティの向上: 制作者と閲覧者の双方が、それぞれの役割においてシステムを円滑に操作でき、コンテンツの制作から配信、そして鑑賞に至る一連のプロセス(エコシステム)が、技術的な破綻や遅延なく統合的に機能する。これにより、本システムが単なる実験的なプロトタイプに留まらず、実用的な展示プラットフォームとしての持続可能性と堅牢性を有していることが実証される。

\section{アンケート}

本実験における定量的評価指標として、システムユーザビリティスケール(System Usability Scale、以下SUS)を採用した。SUSは、Brooke(1996)によって考案されたユーザビリティ評価のための標準的なアンケート手法であり、システムの有効性、効率性、満足度を包括的に測定することが可能である。

本研究においてSUSを選択した理由は、以下の2点である。第一に、SUSは技術的なシステムから日常的な製品まで幅広い対象に適用可能であり、信頼性と妥当性が広く認められている点である。第二に、わずか10項目の質問で構成されており、被験者への負担を最小限に抑えつつ、比較可能なスコア(0点から100点)を算出できる点である。一般的に、SUSスコアが68点以上であれば平均以上のユーザビリティを有していると判断される。

質問項目は5段階のリッカート尺度(1: 全くそう思わない 〜 5: 強くそう思う)で回答を求めた。なお、質問文中の「システム」という単語は、被験者の役割に応じて「AR制作ツール(ARShowNode)」または「AR閲覧アプリ(ARShow)」と読み替えるよう教示を行った。

[表3: 実験で使用したSUS質問項目一覧] (ここにSUSの10項目の質問内容、および回答用スケールの例を示す表を挿入)

\section{被験者の募集}

本実験の被験者は、Googleフォームを用いた公募により選定した。実験の性質上、システムの「制作側」と「閲覧側」双方の視点が必要となるため、計4名の被験者を採用し、それぞれ以下の役割を割り当てた。

\subsection{制作者}

制作者(Creator)役:2名

選定条件:Unityエンジンの基本的な操作(エディタ操作、C\#スクリプトの理解)に習熟していること。

理由:提案システム「ARShowNode」はUnity開発者を対象としたツールキットであるため、一定の開発スキルを有するユーザーによる評価が不可欠であるため。

\subsection{閲覧者}

閲覧者(Viewer)役:2名

選定条件:特段の技術的背景は問わないが、VR/ARデバイスの使用に抵抗がないこと。

理由:一般の美術館や展示会への来場者を想定し、専門知識を持たないユーザーでも直感的に操作可能かを検証するため。

このように役割を分担することで、実際の展示会における「キュレーター(作品提供者)」と「オーディエンス(鑑賞者)」の関係性を模倣し、システム全体のエコシステムとしての妥当性を検証することを意図した。

[表4: 被験者の属性と経験年数] (被験者A〜Dの年齢、性別、Unity使用歴、VR/AR体験頻度などをまとめた属性表を挿入)

\section{提案システム操作方式の紹介}

\subsection{制作者(Creator)への操作説明}

制作者には、Unityプロジェクト「ARShowNode」および本研究で実装した専用エディタ拡張「ARShowTool」の使用方法を説明した。制作者は、Unityエディタ上で3Dモデルの配置やC\#スクリプトの記述(HybridCLR対応)を行い、展品(Node)としての体裁を整えた後、以下の6つの手順を順次実行することでコンテンツの配信が完了することを教授した。

1、DLLのコンパイル (CompileDll): HybridCLRのメニューからCompileDllを実行し、アクティブなビルドターゲット(Android/Quest)に対応したホットアップデート用DLLを出力する。

2、コード生成 (Generate All): 続いてHybridCLRメニューからGenerate(All)を実行し、AOTメタデータおよびC\#とC++間のブリッジコードを生成する。これにより、アプリ本体を更新することなくロジックを配信可能な状態にする。

3、アセンブリの配置 (Move DLLs): ARShowToolメニューのMove DLLsを実行し、生成されたDLLファイルをUnityプロジェクト内のAssetsディレクトリへ自動的に複製する。

4、アセットバンドルの構築 (Build AssetBundles): ARShowToolメニューのBuild AssetBundlesを実行する。事前に設定されたラベル(例: node0, node1)に基づき、Prefab、リソース、および手順3で配置したDLLを含んだAssetBundleファイルを生成する。

5、QRコードの生成 (Generate QR): ARShowToolメニューのGenerate QRを実行し、各AssetBundleの識別子情報を格納したQRコード画像を生成する。制作者はこの画像を保存し、閲覧者に提示する準備を行う。

6、サーバへの配信 (Upload to Server): 最後にARShowToolメニューのUpload to Serverを実行する。生成されたAssetBundle群が、稼働中の静的ファイルサーバの公開ディレクトリへ一括転送され、配信可能な状態となる。

\subsection{閲覧者(Viewer)への操作説明}

閲覧者には、Meta Quest 3にインストールされた「ARShow」アプリの操作方法を説明した。アプリはHMDのパススルー機能を用いたARモードで動作し、以下の手順で鑑賞を行う旨を伝えた。

1、スキャンモードの入り: アプリ起動後、空間上に表示されるUIパネルの「ScanQr」ボタンを指で押下(Poke操作)し、QRコード認識待機モードへ移行する。

2、QRコードの検出: パススルー映像越しに、制作者から提示されたQRコードに視線を向ける。ZXingライブラリによりコードが認識されると、自動的にスキャンモードが終了し、コンテンツのロード処理が開始される。

3、ダウンロードと展開: サーバからAssetBundleのダウンロードが開始され、空間上のプログレスバーに進捗が表示される。完了後、HybridCLRによるDLLのロードとEntryポイントの実行が自動的に行われ、目の前に展品が出現する。

4、インタラクション: 出現した展品に対し、ハンドトラッキングを用いた直感的な操作が可能であることを説明した。具体的には、オブジェクトを直接掴んで移動させる「Grab」、遠くの対象を指し示す「Ray」、ボタンを押す「Poke」、および音声コマンドによる操作を実演と共にレクチャーした。

\section{実験の流れ}

本実験は、実験室環境にて実施した。ハードウェアリソースの制約(開発用PC 1台、Meta Quest 3 1台)を考慮し、被験者を2つのグループに分け、以下の手順で順次実験を行った。

なお、実験の複雑度を制御し、純粋に「提案ツールの操作性」と「鑑賞体験」を評価するため、制作者は新規プロジェクトの立ち上げを行うのではなく、あらかじめ必要な設定(SDK導入、HybridCLR設定済み)が完了している「ARShowNode」プロジェクトを使用した。また、コンテンツ配信用の静的ファイルサーバは実験者が事前に起動し、制作者がサーバ管理を行う必要はないものとした。また、各制作者の実操作前に、前回の実験データによる干渉を防ぐため、ARShowNodeプロジェクト内の生成物(AssetBundles、QRコード画像)およびサーバ上のファイルを完全に削除し、初期状態へのリセットを行った。

\subsection{実験プロトコル}

\subsection{グループ1(制作者1 + 閲覧者1・2)}

制作フェーズ: 制作者1は、ARShowToolを用いて2つの異なる展品「Node0(解説付き文化財)」および「Node1(映像展示)」のビルドからアップロードまでを行う。生成された2つのQRコードを印刷または画面表示により閲覧者に提供する。

閲覧フェーズ: 閲覧者1がQuest 3を装着し、Node0、Node1の順にQRコードをスキャンして体験を行う。体験終了後、デバイスを閲覧者2に交代し、同様にNode0、Node1の体験を行う。

\subsection{グループ2(制作者2 + 閲覧者1・2)}

制作フェーズ: 制作者2は、同様の手順で「Node0」および「Node2(静的3Dモデル)」のビルドとアップロードを行う。生成されたQRコードを閲覧者に提供する。

閲覧フェーズ: グループ1と同様に、閲覧者1、閲覧者2の順でQuest 3を装着し、Node0およびNode2の体験を行う。

全ての体験が終了した後、4名の被験者はGoogleフォームにてSUSアンケートおよび自由記述によるフィードバックへの回答を行った。

[図8: 実験フローチャート] (事前準備からグループ分け、制作、閲覧、アンケート回答に至る一連の流れを示すフロー図を挿入)

\section{実験結果}

4名の被験者から得られたSUSアンケートの回答を集計し、SUSスコア算出定義に基づき0点から100点のスコアに換算した。

結果の概要は以下の通りである。全体の平均スコアは XX.X点 であり、一般的な平均基準とされる68点を大きく上回る結果となった。役割別に見ると、閲覧者(Viewer)の平均スコアは XX.X点 と極めて高く、制作者(Creator)の平均スコアは XX.X点 であった。

自由記述によるフィードバックにおいては、各役割の体験を反映した具体的な意見が得られた。 閲覧者からは、「QRコードを見るだけで体験が始まる点が非常にスムーズだった」「HMDをつけたまま複雑なメニュー操作をしなくて済むのが良い」といった、操作の直感性と身体的負荷の少なさを評価する肯定的な意見が多く寄せられた。 一方、制作者からは、「ツールを使えばワンクリックでアップロードまで完了するのは非常に便利である」というワークフローの効率性を評価する声があった一方で、「エラーが出た際のデバッグが難しい」「実機とエディタの挙動の違いに戸惑った」といった、ホットアップデート開発プロセス特有の課題も指摘された。

[表5: SUSスコア集計結果] (被験者ごとの内訳、役割別平均、全体平均、標準偏差を示す表を挿入)

[図9: 質問項目別平均スコア] (SUSの10項目ごとの平均点を棒グラフで示し、どの項目が高評価/低評価であったかを可視化する図を挿入)

\section{考察}

本節では、アンケート結果および実験中の観察に基づき、提案システムの評価について考察する。

制作者によるSUSスコアが基準点を上回ったこと、および「ワンクリックでの配信」に対する肯定的評価は仮説1を支持する結果である。特に、Unityエディタ上で完結するツールキット(ARShowTool)の提供により、スクリプトのコンパイルからアセットバンドルの生成、サーバ転送までの一連の作業が自動化された点は、従来のアプリ本体の再ビルドを要する手法と比較して、運用コストと時間の著しい低減を実現しているといえる。しかし、スコアが閲覧者に比べてやや低かった要因として、HybridCLRを用いた開発特有の制約(AOTメタデータの管理や、実機でのみ発生するランタイムエラーへの対処など)が、従来のUnity開発フローとは異なる学習コストを要したためと考えられる。したがって、仮説1の有効性は示されたものの、実用化に向けてはツールのUI改善やエラーログの可視化機能の拡充など、開発者体験(DX)の向上が今後の課題である。

閲覧者による極めて高いSUSスコアと、「スムーズな開始」に関するフィードバックは仮説2を支持している。本システムが採用したQRコードをトリガーとする手法は、HMD特有のコントローラ操作やUIに不慣れなユーザーに対しても、物理空間と情報空間をシームレスに接続する直感的なメタファーとして機能したことが確認された。これにより、ユーザーは複雑な導入手順を学習することなく、即座にAR体験へ没入することが可能となり、本システムが高いアクセシビリティを有していることが実証された。

制作者1から制作者2へと配信者が交代し、サーバ上のコンテンツが完全に入れ替わった際も、閲覧者側のアプリは再起動することなく、新しいQRコードを読み込むだけで即座に最新のコンテンツに対応できた。また、LAN環境下での動的ロードにおいても遅延やクラッシュ等の重大な不具合は発生しなかった。これらの事実は、制作者と閲覧者を繋ぐエコシステムが技術的な破綻なく統合的に機能していることを示しており、仮説3を裏付けるものである。提案システムは、展示内容の頻繁な更新や複数作家による共同展示といった実際の運用シナリオにおいても、十分な堅牢性と再現性を備えた実用的なプラットフォームであると結論付けられる。

以上の検証より、本研究で提案した「ARShowNode」および「ARShow」システムは、AR展示における制作者と鑑賞者の双方に対し、従来のアプリ配布型モデルが抱えていた課題を解決する有効なソリューションであることが確認された。今後は、被験者数を増やした大規模な実験や、インターネット経由での遠隔配信実験を行い、さらなる検証とシステムの洗練を進める必要がある。

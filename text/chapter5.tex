\chapter{評価方針}

本章では、前章で提案した「ARShowNode」および「ARShow」から成る AR 展示プラットフォームの実用性と有効性を検証するための実験計画について述べる。本実験では、実際の展示運用を模した一連のワークフローを被験者に実施させ、システムユーザビリティの観点から定量的な評価を行うとともに、自由記述による定性的なフィードバックを収集し、システムの有用性および改善点を明らかにすることを目的とする。

\section{実験仮説}

本研究は、HMD を用いた AR 展示において、制作者(キュレーター)のコンテンツ更新負荷を軽減し、かつ閲覧者(鑑賞者)に対して直感的で没入感のある鑑賞体験を提供することを目的としている。この目的に基づき、本実験では以下の三つの仮説を立証することを試みる。

\subsection{仮説 1: 制作プロセスの効率化}

提案システムが提供する「ARShowTool」およびホットアップデート技術を用いることで、制作者はアプリケーション本体の再ビルドを行うことなく、短時間かつ低コストで複雑なロジックを含むARコンテンツの配信が可能である。これにより、従来のアプリ配布型展示と比較して運用コストが著しく低減される。

\subsection{仮説 2: 鑑賞体験のアクセシビリティ向上}

QR コードを起点とした物理空間と情報空間のシームレスな接続に加え、ハンドトラッキングによる身体的な操作(把持や指差し)および音声コマンドによる自然言語入力を統合することで、HMD 特有のコントローラ操作を排除した直感的な操作体系を実現する。このマルチモーダルなインタラクション設計により、鑑賞者は複雑なメニュー操作や事前の導入手順を学習することなく、自然な挙動で AR 体験を開始および操作できる高いアクセシビリティを有する。

\subsection{仮説 3: システム全体の統合的ユーザビリティ}

制作者と閲覧者の双方が、それぞれの役割においてシステムを円滑に操作可能であり、Unity ツールによるコンテンツ生成からサーバ配信、そしてクライアントアプリによる鑑賞に至る一連のプロセスが、技術的な破綻や遅延なく統合的に機能する。これにより、本システムが単なるプロトタイプに留まらず、実用的な展示プラットフォームとしての実現可能性を有していることが示される。

\section{評価手法}

\subsection{定量的評価(SUS)}

本実験における定量的評価指標として、System Usability Scale(以下、SUS)を採用した。SUS は、Brooke(1996)によって考案されたユーザビリティ評価のための標準的なアンケート手法であり、システムの有効性、効率性、満足度を包括的に測定することが可能である。

SUS を採用した理由は以下の2点である。

\begin{itemize}
    \item \textbf{第一}: 技術的なシステムから一般的な製品まで幅広い対象に適用可能であり、学術的にも信頼性と妥当性が広く認められている点である。
    \item \textbf{第二}: 10項目の質問で構成されており、被験者への負担を最小限に抑えつつ、比較可能なスコアを算出できる点である。
\end{itemize}

質問項目は 5 段階のリッカート尺度(1: 全くそう思わない 〜 5: 強くそう思う)で回答を求める。なお、質問文中の「システム」という用語は、被験者の役割に応じて「AR 制作ツール(ARShowNode)」または「AR 閲覧アプリ(ARShow)」と読み替えるよう教示を行う。本実験で使用する SUS の質問項目を表\ref{table:sus_questions}に示す。

\begin{table}[htbp]
  \caption[SUSアンケート項目]{本実験で使用する System Usability Scale(SUS)の質問項目}
  \label{table:sus_questions}
  \centering
  \begin{tabular}{rp{110mm}}
  \hline
  \multicolumn{1}{c}{No.} & \multicolumn{1}{c}{質問内容} \\ \hline \hline
  1 & このシステムを頻繁に使用したいと思うか。 \\ 
  2 & このシステムは必要以上に複雑だと感じるか。 \\ 
  3 & このシステムは使いやすいと感じるか。 \\ 
  4 & このシステムを使用するために、技術的なサポートが必要だと感じるか。 \\ 
  5 & このシステムの様々な機能はよく統合されていると感じるか。 \\ 
  6 & このシステムには一貫性がない(矛盾が多い)と感じるか。 \\ 
  7 & 多くの人々がこのシステムの使い方をすぐに習得できると思うか。 \\ 
  8 & このシステムは操作が非常に煩雑(面倒)だと感じるか。 \\ 
  9 & このシステムを使っていて安心感(自信)を持てるか。 \\ 
  10 & このシステムを使い始める前に、多くのことを学ぶ必要があると感じるか。 \\ \hline
  \end{tabular}
\end{table}

\subsection{定性的評価}

SUSによる定量評価に加え、自由記述形式のアンケートを実施する。ここでは、操作中に感じた具体的な困難点、システムの挙動に関する違和感、および将来的な機能要望について記述を求め、SUS 数値には現れないユーザビリティの課題を抽出する。

\section{被験者の構成}

本実験の被験者は、Google フォームを用いた公募により選定する。実験の性質上、システムの「制作側」と「閲覧側」双方の視点が必要となるため、計 4 名の被験者を採用し、それぞれ以下の役割を割り当てる。

\subsection{制作者(Creator)}
\begin{itemize}
    \item \textbf{人数}: 2 名
    \item \textbf{選定条件}: Unity エンジンの基本的な操作(エディタ操作、C\# スクリプトの理解)に習熟していること。
    \item \textbf{選定理由}: 提案システム「ARShowNode」は Unity 開発者を対象としたツールキットであるため、一定の開発スキルを有するユーザーによる専門的な評価が不可欠であるため。
\end{itemize}

\subsection{閲覧者(Viewer)}
\begin{itemize}
    \item \textbf{人数}: 2 名
    \item \textbf{選定条件}: 特段の技術的背景は問わないが、HMD デバイスの使用に抵抗がないこと。
    \item \textbf{選定理由}: 一般の美術館や展示会への来場者を想定し、専門知識を持たないユーザーでも直感的に操作可能かを検証するため。
\end{itemize}

このように役割を分担することで、実際の展示会における「キュレーター(作品提供者)」と「オーディエンス(鑑賞者)」の関係性を模倣し、システム全体のエコシステムとしての妥当性を検証することを意図する。

\section{操作説明}

実験に先立ち、各被験者に対して以下の通り操作説明(教示)を行う。

\subsection{制作者への操作説明}

制作者役には、Unity プロジェクト「ARShowNode」および本研究で実装した「ARShowTool」の使用方法を説明する。以下の六つの手順を順次実行することで、コンテンツの配信が完了することを教授する。

\begin{enumerate}
    \item \textbf{DLL のコンパイル(CompileDll)}: HybridCLR メニューから「CompileDll」を実行し、ターゲットプラットフォーム(Android)に対応したホットアップデート用 DLL を出力する。
    \item \textbf{コード生成(GenerateAll)}: 続いて HybridCLR メニューから「GenerateAll」を実行し、AOT メタデータおよびブリッジコードを生成する。これにより、アプリ本体を更新することなくロジック配信を可能にする。
    \item \textbf{アセンブリの配置(MoveDLLs)}: ARShowTool メニューの「MoveDLLs」を実行し、生成された DLL ファイルを Unity プロジェクト内の Assets ディレクトリへ自動的に複製する。
    \item \textbf{アセットバンドルの構築(BuildAssetBundles)}: ARShowTool メニューの「BuildAssetBundles」を実行する。事前に設定されたラベル(例: node0, node1)に基づき、リソースと DLL を含んだ AssetBundle ファイルを生成する。
    \item \textbf{サーバへの配信(UploadToServer)}: 最後に ARShowTool メニューの「UploadToServer」を実行する。生成された AssetBundle 群が、稼働中の静的ファイルサーバへ一括転送され、配信状態となる。
\end{enumerate}

\subsection{閲覧者への操作説明}
閲覧者役には、Meta Quest 3 にインストールされた「ARShow」アプリの操作方法を説明する。アプリはパススルー機能を用いたARモードで動作し、以下の手順で鑑賞を行う旨を伝える。

\begin{enumerate}
    \item \textbf{スキャンモードの開始}: アプリ起動後、空間上に表示される「ScanQR」ボタンを押下(Poke)し、QRコード認識待機モードへ移行する。
    \item \textbf{QR コードの検出}: パススルー映像越しに、制作者から提示された QR コードに視線を向ける。認識されると自動的にロード処理が開始される。
    \item \textbf{ダウンロードと展開}: サーバからの AssetBundle ダウンロード、HybridCLR による DLL ロード、および Entry ポイントの実行が自動的に行われ、目の前に展品が出現する。
    \item \textbf{インタラクション}: 出現した展品に対し、ハンドトラッキングを用いた操作(Grab、Ray、Poke、音声コマンド)を行い、鑑賞する。
\end{enumerate}

\section{実験手順}

本実験では、実験における外的要因を排除し、純粋に提案ツールの有用性と AR 展品の鑑賞体験を評価するため、制作者による「新規 Unity プロジェクトの作成」工程は省略し、ホスト(実験実施者)が以下の環境準備を行うものとする。

\begin{itemize}
    \item \textbf{事前準備}: 開発用 PC に「ARShowNode」プロジェクトをセットアップし、静的ファイルサーバを起動する。
    \item \textbf{環境の初期化}: 各実験セットの開始前に、前回のデータによる干渉を防ぐため、生成物(AssetBundles、QR コード)およびサーバ上のファイルを完全に削除し、初期状態にリセットする。
\end{itemize}

ハードウェアリソースの制約(PC 1 台、HMD 1 台)を考慮し、被験者を二つのグループに分け、図\ref{fig:experiment_flow}に示す流れで順次実験を行う。

% 実験流れ図
\begin{figure}[htbp]
\begin{center}
\includegraphics[width=80mm]{./figs/評価方針/実験流れ図.png}
\caption{本実験の実施フロー [5].}
\label{fig:experiment_flow}
\end{center}
\end{figure}

\subsection{グループ A(制作者 1、閲覧者 1 2)}

\begin{enumerate}
    \item \textbf{制作フェーズ}: 制作者 1 は、ARShowTool を用いて二つの異なる展品「Node0(解説付きキャンバス)」および「Node1(映像展示)」のビルドからアップロードまでを行う。生成された二つの QR コードを画面表示により閲覧者に提供する。
    \item \textbf{閲覧フェーズ}: 閲覧者 1 が Quest 3 を装着し、Node0、Node1 の順に QR コードをスキャンして体験を行う。体験終了後、HMD を閲覧者 2 に渡し、同様に体験を行う。
\end{enumerate}

\subsection{グループ B(制作者 2、閲覧者 1 2)}

\begin{enumerate}
    \item \textbf{制作フェーズ}: 制作者 2 は、同様の手順で「Node0」および「Node2(静的3Dモデル)」のビルドとアップロードを行う。これにより、異なるデータタイプの処理に対する安定性を検証する。
    \item \textbf{閲覧フェーズ}: グループ A と同様に、閲覧者 1、閲覧者 2 の順で Quest 3 を装着し、Node0 および Node2 の体験を行う。
\end{enumerate}

全ての体験が終了した後、4 名の被験者は Google フォームにて SUS アンケートおよび自由記述によるフィードバックへの回答を行う。
\chapter*{概要}

本論文では、3D モデルと QR コードを統合し、ヘッドマウントディスプレイ(HMD)を通じて拡張現実(AR)空間での高度な可視化を実現するプレゼンテーションシステム「ARShow」の開発、設計、およびその有用性について詳述する。AR 技術は現在、教育、マーケティング、娯楽など多岐にわたる分野で活用されているが、アクセシビリティとインタラクティブ性の高度な両立が依然として課題となっている 。本研究では、制作者向けのキュレーションサイドと閲覧者向けの鑑賞サイドからなる双方向的なプラットフォームを提案し、Unity エンジンを基盤とした技術的仕様、アセットのインポート・エクスポート、および 3D モデルの可視化プロセスについて論じる 。

先行研究である AR MUSE は、Android 端末を用いて 3D モデルと QR コードを紐付ける安価かつユーザーフレンドリーな解決策を提示した 。これに対し、本研究の独創性は、出力デバイスとして HMD を採用することで没入感を飛躍的に向上させ、さらに 3D モデルと共に「動的スクリプト」を同時配信可能にした点にある。これにより、従来の静的な展示に留まらない、複雑な挙動を伴うインタラクティブな AR 表現を閲覧者に提供することが可能となった。

システムの評価にあたっては、Unity 開発経験を有する制作者および一般の閲覧者を対象とした被験者実験を行い、システムユーザビリティ(SUS)の観点から定量的・定性的な分析を実施した。その結果、ARShow は高度な AR 展示の構築において、アクセシビリティに優れ、かつ表現力豊かな実用的ソリューションであることが実証された。本研究の成果は、3D モデルを活用するクリエイターに対し、AR をより身近で強力なプレゼンテーション手法として確立させるものである 。


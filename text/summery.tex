\chapter{まとめ}

\section{総括}

本研究では、独立系キュレーターによる AR 空間芸術展示の普及に伴い顕在化した、展示運用におけるコミュニケーションコストの増大と、コンテンツ更新プロセスの硬直性という課題に着目した。 

従来のアプリケーション配布型の展示手法では、鑑賞者に対する導入支援の負担が大きく、また展示ロジックの些細な修正であってもアプリケーション全体の再ビルドと再配布を要するため、柔軟な展示運用が困難であった。そこで本研究では、HMD と QR コードを活用し、制作者(キュレーター)と閲覧者(鑑賞者)をシームレスに接続する中間プラットフォームを提案および開発した。

本研究の主な成果は以下の通りである。

\begin{itemize}
    \item \textbf{第一}: Unity の AssetBundle 機能と HybridCLR フレームワークを統合することで、従来困難とされていた C\# スクリプトのホットアップデートを実現するアーキテクチャを構築した。これにより、制作者はアプリケーション本体を更新することなく、3D モデルのみならずインタラクションロジックを含めた展示コンテンツを動的に配信かつ更新することが可能となった。また、制作者向けに実装した Unity 拡張ツール「ARShowTool」により、DLL のビルドから AssetBundle の生成、サーバへのアップロードまでの一連の工程を自動化し、制作プロセスを大幅に効率化した。
    \item \textbf{第二}: Meta Quest 3 のパススルー機能とハンドトラッキング、および音声認識技術(Wit.ai)を組み合わせた、直感的かつ没入感の高い鑑賞環境を実現した。鑑賞者は会場の QR コードをスキャンするだけで、必要なコンテンツをオンデマンドで取得し展開できるため、複雑なアプリ操作や事前設定を不要とした。また、開発したプロトタイプにおいて、複合 UI、映像、3D モデルという異なるメディア形式の展示が可能であることを示し、システムの汎用性を確認した。
    \item \textbf{第三}: 制作者と閲覧者の役割を分担した被験者実験を通じて、SUS による定量的評価および自由記述による定性的評価を行う方針を定めた。本システムは、LAN(構内通信網)環境下における動作検証において、意図通りのコンテンツ配信と動的ロードが機能することを確認できており、実用的な展示プラットフォームとしての基盤確立を達成したといえる。
\end{itemize}

\section{今後の課題}

本研究で提案したシステムの実用化に向けた今後の課題として、以下の点が挙げられる。

\subsection{評価実験の実施と検証}

本論文執筆時点において、前章で設計された評価実験はまだ未実施である。そのため、実際の被験者を対象とした実験を行い、以下の仮説を検証する必要がある。具体的には、制作者にとっての「制作と更新プロセスの効率化」、鑑賞者にとっての「アクセシビリティの向上」、そしてシステム全体の「統合的ユーザビリティ」である。定量的と定性的なデータを収集し分析することが急務である。

\subsection{コンテンツの安全性審査}

本システムは、制作者が作成した任意のスクリプトやアセットを実行時にロードする仕組みであるため、配信されるコンテンツの品質管理が重要な課題となる。現在、アップロードされた AssetBundle に対する審査機構は実装されていない。しかし、一般公開を想定した場合、公序良俗に反するコンテンツや、激しい光の点滅など鑑賞者の健康を害する恐れのある演出(光過敏性発作の誘発リスク等)が含まれる可能性がある。したがって、サーバへのアップロード時に自動でメタデータを解析するフィルタリング機能や、運営者による事前承認プロセスの導入など、コンテンツの安全性と健全性を担保する仕組みの検討が必要である。

\subsection{セキュリティの強化}

HybridCLR を用いたホットアップデートは強力な機能である反面、セキュリティリスクを伴う。悪意のある制作者が、デバイス内の個人情報にアクセスするスクリプトや、システムをクラッシュさせるコードを AssetBundle に混入させるリスクが考えられる。現状のシステムは信頼できる制作者(Closed 環境)を前提としているが、プラットフォームを開放するにあたっては、実行可能な C\# API の制限(サンドボックス化)や、DLL に対する電子署名による改ざん検知機能の実装が不可欠である。

\subsection{クラウド環境への移行とスケーラビリティ}

現在の実装は LAN(構内通信網)内での運用を前提とした静的ファイルサーバ構成であるが、大規模な展示会や遠隔地からのアクセスに対応するためには、クラウドインフラへの移行が必要である。AWS や Cloudflare 等の CDN(コンテンツデリバリネットワーク)を活用し、地理的に分散したアクセスに対しても高速に AssetBundle を配信できるアーキテクチャへの拡張が求められる。また、同時に多数の鑑賞者が QR コードをスキャンした際のサーバ負荷分散や、ネットワーク帯域の最適化についても検討の余地がある。
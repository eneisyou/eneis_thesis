\chapter*{Abstruct}

This thesis discusses the development, design, and efficacy of ARShow, a presentation system for Head-Mounted Displays (HMDs) that facilitates the integration of 3D models with QR codes to enable advanced augmented reality (AR) visualization. While AR technology is currently employed across diverse fields such as education, marketing, and entertainment, the simultaneous achievement of accessibility and high-level interactivity remains a persistent challenge. This research proposes a bidirectional platform comprising a "curation side" for creators and a "viewer side" for audiences, detailing technical specifications based on the Unity engine as well as the processes for asset importation, exportation, and 3D model visualization.

The predecessor research, AR MUSE, offered an affordable and user-friendly solution for linking 3D models to QR codes via Android devices. In contrast, the novelty of the present study lies in the adoption of HMDs as the output medium to drastically enhance immersion, as well as the implementation of simultaneous distribution for "dynamic scripts" alongside 3D models. This allows for the delivery of interactive AR experiences involving complex behaviors that extend beyond conventional static exhibits.

The system was evaluated through subject testing involving both creators with Unity development experience and general viewers, with quantitative and qualitative analyses performed using the System Usability Scale (SUS). The results indicate that ARShow serves as an accessible, highly expressive, and practical solution for the construction of advanced AR exhibitions. This work aims to establish AR as a more accessible and powerful presentation methodology for creators utilizing 3D models globally.

\chapter*{Abstruct}

This paper details the development, design, and utility of "ARShowNode" and "ARShow," presentation systems that integrate 3D models with QR codes to achieve advanced visualization in AR (Augmented Reality) space via HMDs (Head Mounted Displays). While AR technology is currently utilized in diverse fields such as education, marketing, and entertainment, achieving a high level of compatibility between accessibility and interactivity remains a challenge. This study proposes a bidirectional platform consisting of the "ARShowNode" project for creators and the "ARShow" project for viewers, and discusses the technical specifications based on the Unity engine, the asset upload and download mechanisms, and the visualization process of 3D models.

Previous research, such as "AR MUSE," presented inexpensive and user-friendly solutions that associate 3D models with QR codes using Android devices. In contrast, the novelty of this research lies in significantly enhancing immersion by adopting HMDs as output devices and enabling the simultaneous distribution and execution of dynamic scripts alongside 3D models. This capability allows for the provision of interactive AR expressions involving complex behaviors to viewers, surpassing traditional static exhibitions.

To evaluate the system, subject experiments were conducted targeting creators with Unity development experience and general viewers, performing both quantitative and qualitative analyses from the perspective of system usability. The results demonstrate that the system proposed in this study is a practical solution offering excellent accessibility and rich expressiveness for the construction of advanced AR exhibitions.
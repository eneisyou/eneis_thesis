\begin{thebibliography}{99}

\bibitem{ref:journal}
Ohlei, A., Schumacher, T., \& Herczeg, M. (2020). An Augmented Reality Tour Creator for Museums with Dynamic Asset Collections. In Augmented Reality, Virtual Reality, and Computer Graphics (LNCS 12243, pp. 15-31). Springer. 
\bibitem{ref:journal}
Duanmu, Q., Dai, T., Cai, Y., \& Herman, J. (2023). AR MUSE: Designing and Implementing a Solution for Accessible Augmented Reality Exhibition. Worcester Polytechnic Institute. 
\bibitem{ref:journal}
Bekele, M.K. Clouds-Based Collaborative and Multi-Modal Mixed Reality for Virtual Heritage. Heritage 2021, 4, 1447-1459. 
\bibitem{ref:journal}
Kidman, B. A Platform for in-Situ Creation of Markerless, Location-Based Augmented Reality Content. Master's Thesis, Dartmouth College, 2023. 
\bibitem{ref:journal}
飛田博章, 渡辺光太郎, 山川美咲, 小島瑛里子. 拡張現実を用いた作品に対するコメントを共有することによる対話型美術鑑賞の支援. 日本・美術による学び学会誌, 第 6 巻, 第 1 号.
\bibitem{ref:journal}
Leandro Soares Guedes, Luiz André Marques, and Gabriellen Vitório. "Enhancing interaction and accessibility in museums and exhibitions with Augmented Reality and Screen Readers." Università della Svizzera italiana / Federal Institute of Mato Grosso do Sul. 
\bibitem{ref:journal}
伏田昌弘, 赤羽亨. "画像マーカーベースの AR を用いた音声ガイドの試作." 情報処理学会 インタラクション 2024, IA-16, pp. 253-256, 2024. 
\bibitem{ref:journal}
Kyriakou, P. and Hermon, S.: Can I touch this? Using Natural Interaction in a Museum Augmented Reality System, Digital Applications in Archaeology and Cultural Heritage, Vol. 12, e00088 (2018). 
\bibitem{ref:journal}
Liu, Y., Spierling, U., Rau, L. and Dörner, R.: Handheld vs. Head-Mounted AR Interaction Patterns for Museums or Guided Tours, Intelligent Technologies for Interactive Entertainment (INTETAIN 2020), LNICST 377, pp. 229-242 (2021). 
\bibitem{ref:journal}
Liu, Y., Bitter, J.L. and Spierling, U.: Evaluating Interaction Challenges of Head-Mounted Device-based Augmented Reality Applications for First-time Users at Museums and Exhibitions. 
\bibitem{ref:journal}
Ramy Hammady, Minhua Ma, Ziad AL-Kalha, and Carl Strathearn. A framework for constructing and evaluating the role of MR as a holographic virtual guide in museums. Virtual Reality, Vol. 25, pp. 1-25, 2021. 
\bibitem{ref:journal}
井上道哉, 長澤可也. 綾瀬市埋蔵文化財の VR 、 AR コンテンツ化による地域活性化 湘南工科大学紀要, Vol. 55, No. 1, pp. 41-47, 2021. 
\bibitem{ref:journal}
Weiting Hou: Augmented Reality Museum Visiting Application based on the Microsoft HoloLens, Journal of Physics: Conference Series, Vol. 1237, 052018, 2019. 
\bibitem{ref:journal}
赤嶺有平: 拡張現実を用いた博物館における双方向メディア型ガイダンスシステムの開発. 科学研究費助成事業 研究成果報告書, 課題番号 19K13045, 2021. 
\bibitem{ref:journal}
星野浩司: AR 型遠隔学習支援システム「 AI Aquarium 」の開発. 九州産業大学芸術学部研究報告, 第 56 巻, pp. 38-41, 2024. 
\bibitem{ref:journal}
Robert Lee Seligmann. "Web-based Client for Remote Rendered Virtual Reality". Master's Thesis, Aalto University, 2020. 
\bibitem{ref:journal}
Viktor Kelkkanen, Markus Fiedler, and David Lindero. "Synchronous Remote Rendering for VR". International Journal of Computer Games Technology, Vol. 2021, Article ID 6676644, 2021. 
\bibitem{ref:journal}
Noor Hammad, Thomas Eiszler, Robert Gazda, John Cartmell, Erik Harpstead, and Jessica Hammer. "V-Light: Leveraging Edge Computing For The Design of Mobile Augmented Reality Games". In Proceedings of the 18th International Conference on the Foundations of Digital Games (FDG '23), 2023.


% =====================
% 参考文献 書き方例

% \bibitem{ref-related-work}
% 論文の読み方・書き方, 金森 由博, \url{http://kanamori.cs.tsukuba.ac.jp/docs/how_to_read_and_write_papers.html}, 2021/09/28.
% \bibitem{ref-jsps} 
% 研究者のみなさまへ~責任ある研究活動を目指して~, 国立研究開発法人科学技術振興機構, \url{https://www.jst.go.jp/researchintegrity/shiryo/pamph_for_researcher.pdf}, 2020. 

% \bibitem{ref-ieice} 
% 和文論文誌投稿のしおり, 電子情報通信学会, \url{https://www.ieice.org/jpn/shiori/iss_2.html#2.6}, 2021/09/28.

% \bibitem{ref-ipsj} 
% 論文誌ジャーナル原稿執筆案内, 情報処理学会, \url{https://www.ipsj.or.jp/journal/submit/ronbun_j_prms.html}, 2021/09/28.

% \bibitem{ref:journal}
% Almuzaini, K.K., Joshi, S., Ojo, S. et al.,
%    "Optimization of the operational state's routing for mobile wireless sensor networks," 
%    Wireless Networks, pp. 1-15, 2023.

% \bibitem{ref-journal} 
% 著者名, ``論文タイトル,'' 雑誌名, vol, no, page, year.

% \bibitem{ref-journal-ex} 
% 國田 樹, 遠藤聡志, ``学術論文の出典記載例,'' 知能情報学会誌, vol. 3, no. 2, pp.8-13, 2021.

% \bibitem{ref:book}
% 中垣俊之, ``粘菌その驚くべき知性,'' 株式会社PHP 研究所, 東京, 2010.

% \bibitem{ref-book}
% 著者名, ``書籍タイトル,'' (編集者名), 出版社名, 発行都市名, 発行年.

% \bibitem{ref-book-ex}
% 國田樹, ``著書の出典記載例,'' 知能情報出版, 沖縄, 2021.

% \bibitem{ref-proceedings}
% 著者名, ``論文タイトル,'' 学会名もしくは会議名, no.論文番号, ページ, 開催都市名, 開催国名, year. 

% \bibitem{ref-proceedings-ex}
% 國田樹, 遠藤聡志, ``学会論文の出典記載例'' 第2回知能情報国際会議, no.2-1234, pp.1-8, Okinawa, Japan, 2021. 

% \bibitem{ref:web}
% What is cloud cost optimization?, IBM, 
% \url{https://www.ibm.com/blog/what-is-cloud-cost-optimization/}, 2024/01/11.

% \bibitem{ref-web}
% 著者名(サイト管理者と同一の場合は省略可), Webページタイトル, サイト管理者名等, URL(url命令を使用すること), 参照年月日.  

% \bibitem{ref-report1}
% 見延庄太郎,理系のためのレポート・論文完全ナビ,講談社, 2016.

% \bibitem{ref-report2}
% 福地健太郎,理工系のためのよい文章の書き方,翔泳社, 2019.

\end{thebibliography}

\chapter{はじめに}
\section{背景と目的}

近年、博物館や美術館をはじめとする多様な文化施設において、デジタル技術を導入した展示手法が急速に普及している。特に、AR(拡張現実)技術を用いた空間芸術展示は、物理的な制約を超えた情報の提示や、現実空間とデジタルコンテンツが融合する新たな視覚体験を提供する手段として、その重要性を増している。このような動向の中で、特定の組織に属さず、独自のテーマや視点で展覧会を企画と構成する独立系キュレーターの活動が活発化しており、デジタル空間における展示設計の担い手も多様化している。

しかし、ARを用いたデジタル展示の現場では、運用面における大きな課題が浮き彫りとなっている。第一に、展示運営におけるコミュニケーションコストの増大である。AR空間芸術のような専門性の高い展示では、キュレーターが鑑賞者一人ひとりに対し、アプリケーションの導入方法や操作手順を詳細に説明する必要がある。また、鑑賞者側もキュレーターごとに異なる独自の展示手法や操作体系をその都度理解しなければならず、これが円滑な鑑賞体験の障壁となっている。第二に、コンテンツの更新性と配布プロセスの硬直性である。従来のARアプリケーション開発において、特に Unity 等のゲームエンジンを用いたiOS、Android、あるいは Meta Quest 等の HMD 向けビルドでは、セキュリティやプラットフォームの制約上、実行バイナリ自体の更新なしにプログラムの挙動(C\#スクリプト)を変更することは極めて困難であった。このため、展示内容を柔軟かつ即時に反映させることができず、微細な修正であってもアプリケーション全体の再ビルドと再配布を余儀なくされていた。キュレーター自身がこの複雑な開発と更新プロセスに関与することは事実上不可能であり、結果として展示制作と運用の効率性が著しく損なわれている。

関連研究である「AR MUSE」等の既存システムでは、AssetBundle 技術を用いることで 3D モデル等の非コードアセットの動的読み込みを実現しているが、スクリプトロジックの更新には対応していない。そのため、インタラクションの振る舞いは事前にコンパイルされた範囲に限定され、動的な演出の追加やロジックの変更といったニーズに応えることができなかった。加えて、多くの既存システムはスマートフォン等のモバイル端末を対象としており、空間芸術において重要な没入感の提供という点においても課題が残されている。

\begin{figure}[htbp]
\begin{center}
\includegraphics[width=100mm]{./figs/はじめに/ar_muse_viewer_app_main_progress.png}
\caption[対話型美術鑑賞支援システムの構成図]{対話型美術鑑賞支援システムの構成図 [5].}
\label{fig:tobita_config}
\end{center}
\end{figure}

本研究の目的は、上述した課題を解決するため、HMD (ヘッドマウントディスプレイ)と QR コードを活用し、制作者としてのキュレーターと閲覧者としての鑑賞者の間を媒介する中間的なアプリケーション基盤 ARShow を提案および構築することである。

\begin{figure}[htbp]
\begin{center}
\includegraphics[width=100mm]{./figs/はじめに/ARShowシステムアーキテクチャ.png}
\caption[対話型美術鑑賞支援システムの構成図]{対話型美術鑑賞支援システムの構成図 [5].}
\label{fig:tobita_config}
\end{center}
\end{figure}

具体的には、以下の三つの目標を達成することを目指す。

第一に、アプリケーション本体の更新を一切必要とせずに、最新の展示内容とロジックを即時に公開かつ反映できる仕組みの構築である。本研究では、Unity 向けフレームワークである HybridCLR を導入することで、従来困難であったスクリプトレベルでのコードホットアップデートを実現する。これにより、キュレーターは自身の制作した3Dモデルやインタラクションロジックを迅速に配信可能となる。

第二に、展示会場におけるコミュニケーションコストの大幅な削減である。鑑賞者が会場に掲示された QR コードをスキャンするだけで、対応するコンテンツを静的サーバから即座にダウンロードし、実行環境へロードするワークフローを確立する。これにより、煩雑な操作説明やアプリの個別導入を不要とし、鑑賞体験の開始を円滑化する。

第三に、HMDを用いた没入感の高いAR体験の提供である。ビデオシースルー機能を備えた Meta Quest 3 等の最新デバイスを活用し、Meta XR SDKを通じて現実空間とデジタルコンテンツをシームレスに融合させることで、空間芸術としての質を担保した鑑賞体験を実現する。

本研究を通じて、展示内容の更新がアプリケーションの再配布に依存しない柔軟な運用を実現する。キュレーターにとっては表現の自由度と制作効率を高め、鑑賞者にとっては手軽で没入感のある体験を提供し、新たなデジタル展示のプラットフォームを確立することを目指す 。

\section{論文の構成}

本論文の構成は以下の通りである.

第 2 章 基礎概念を述べる.

第 3 章 関連研究を述べる.

第 4 章 提案システムを詳細に説明する.

第 5 章 評価方針を立てる.

第 6 章 総括と今後の課題を述べる.

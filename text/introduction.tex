\chapter{はじめに}
\section{背景と目的}

近年、博物館や美術館をはじめとする多様な文化施設において、デジタル技術を導入した展示手法が急速に普及している。特に、AR(拡張現実)技術を用いた空間芸術展示は、物理的な制約を超えた情報の提示や、現実空間とデジタルコンテンツが融合する新たな視覚体験を提供する手段として、その重要性を増している。

こうした展示環境の変化に伴い、コンテンツの制作および運用を担う主体の役割も変容している。従来の博物館展示において、キュレーター(学芸員)の主たる役割は資料の収集・保存・調査研究であった。しかし、メディアアートやインタラクティブな展示が増加する現代においては、施設空間を活用しつつ、Unity 等のゲームエンジンを用いて自ら AR コンテンツの実装やインタラクション設計を行う、エンジニアリング能力を有したアーティストや展示製作者がその中心となりつつある。そこで本研究では、便宜上こうした「技術的背景を持ち、展示の企画から実装までを担う人物」をキュレーターと定義し、一般的な博物館法に基づく学芸員とは区別して論じるものとする。

これらエンジニアリング能力を持つキュレーターの活動は、展示表現の可能性を広げる一方で、実際の運用現場においては依然として大きな課題を抱えている。

第一の課題は、展示運営におけるコミュニケーションコストの増大である。AR 空間芸術のような専門性の高い展示では、鑑賞者が自身のデバイス等で体験を行う際、キュレーターが一人ひとりに対しアプリケーションの導入方法や操作手順を詳細に説明する必要が生じる場合が多い。鑑賞者にとっても、展示ごとに異なる独自の操作体系をその都度理解することは負担であり、これが円滑な鑑賞体験を阻害する要因となっている。

第二の課題は、コンテンツの更新性と配布プロセスの硬直性である。本研究が対象とするキュレーターは Unity 開発に習熟しているものの、iOS、Android、あるいは Meta Quest 等の HMD 向けビルドにおいては、セキュリティやプラットフォームの制約上、実行バイナリ自体の更新なしにプログラムの挙動(C\# スクリプト)を変更することは極めて困難である。そのため、展示期間中に演出の微調整やロジックの修正が必要になった場合でも、アプリケーション全体の再ビルドと再配布を余儀なくされる。特に展示会場において、Quest 端末等の HMD に対して再インストールを行う作業は、PC 接続や開発者モードでの操作を要し、1 回の更新に多大な時間を要するため、複数台のデバイスを運用する展示現場では現実的ではない。したがって、アプリ本体を更新せずに展示ロジックのみを更新可能なホットアップデート機構の確立は、展示の質と鮮度を維持するために不可欠な要素である。

本研究の目的は、上述した課題を解決するため、HMD (ヘッドマウントディスプレイ)と QR コードを活用し、制作者としてのキュレーターと閲覧者としての鑑賞者の間を媒介する中間的なアプリケーション基盤「ARShow」を提案および構築することである(図\ref{fig:arshow_arch})。

\begin{figure}[htbp]
\begin{center}
\includegraphics[width=100mm]{./figs/はじめに/ARShowシステムアーキテクチャ.png}
\caption[ARShowシステムアーキテクチャ]{ARShowシステムアーキテクチャ.}
\label{fig:arshow_arch}
\end{center}
\end{figure}

具体的には、以下の三つの目標を達成することを目指す。

\begin{itemize}
    \item \textbf{第一}: Unity 向けフレームワークである HybridCLR を導入することで、従来困難であったスクリプトレベルでのコードホットアップデートを実現する。これにより、キュレーターは「ARShowTool」を通じて、自身の制作した 3D モデルや複雑なインタラクションロジックを、再ビルドの手間なく迅速かつ低コストで配信可能な環境を構築する。
    \item \textbf{第二}: QR コードをスキャンするだけで即座に体験を開始できる導入の簡便さに加え、HMD 特有の物理コントローラを排除したハンドトラッキングによる身体的操作や、音声コマンドによる自然言語入力を統合する。これらマルチモーダルなインタラクションの実現により、鑑賞者の学習コストを最小化し、誰もが直感的に参加できる没入感の高い AR 体験を提供する。
    \item \textbf{第三}: 単なる機能の提案に留まらず、Unity 上での制作、サーバへの配信、そしてクライアントアプリでの動的ロードという一連のワークフローが、実際の展示運用に耐えうるシステムとしての有効性とユーザビリティを明らかにする。
\end{itemize}

本研究を通じて、展示内容の更新がアプリケーションの再配布に依存しない柔軟な運用環境を実現する。これにより、開発スキルを持つキュレーターにとっては表現の自由度と制作効率を高め、鑑賞者にとっては手軽で没入感のある体験を提供し、新たなデジタル展示のプラットフォームを確立することを目指す。

\section{論文の構成}

本論文の構成は以下の通りである.

第 2 章 基礎概念を述べる.

第 3 章 関連研究を述べる.

第 4 章 提案システムを詳細に説明する.

第 5 章 評価方針を立てる.

第 6 章 総括と今後の課題を述べる.
